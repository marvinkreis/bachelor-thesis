\begin{frame}
    \bigcenter{Why is automated testing for Scratch difficult?}
\end{frame}

\begin{frame}\frametitle{Why is automated testing for Scratch difficult?}
    Scratch is difficult to interact with in an automated way:
    \begin{itemize}
        \item \textcolor{upfim}{no functions}, that take parameters and return a value
        % \item Scratch is only accessible through its GUI
        \item \textcolor{upfim}{no textual IO}, keyboard / mouse input and graphical output
    \end{itemize}

    \bigskip

    \centering
    \includegraphics[width=.45\textwidth]{scratch-code}
    \hspace{1em}
    \includegraphics[width=.4\textwidth]{scratch-stage}
\end{frame}

\begin{frame}
    \bigcenter{How to test Scratch programs?}
\end{frame}

\begin{frame}[fragile]\frametitle{How to test Scratch programs? Automating IO}
    Approach: Test on a system level by automating Scratch's IO

    \begin{figure}
        \begin{center}
            \tikzset{>=latex,
                   arrow/.style={draw, -{Latex[length=1.5mm, width=1.5mm]}},
                     put/.style={draw, minimum height=0.65cm, minimum width=1.75cm, rounded corners, fill=red!20, text width=2.4cm, text centered},
                      vm/.style={draw, minimum height=1.75cm, minimum width=3.0cm, rounded corners, fill=white},
                     gui/.style={draw, minimum height=2.6cm, minimum width=3.5cm, rounded corners, fill=blue!20},
                 whisker/.style={draw, minimum height=2.6cm, minimum width=3.5cm, rounded corners, fill=green!20},
                     box/.style={draw,text centered, rounded corners}}

            \hspace{2mm}\begin{tikzpicture}[scale=0.9, every node/.style={scale=0.9}]
                \node[box] at (0.0, 4.0) (input)  {\includegraphics[height=.25\textheight]{mouse-keyboard}};
                \node[box] at (5.1, 4.0) (output) {\includegraphics[height=.25\textheight]{scratch-stage}};

                \node[] at (0.0, 5.5) (inputtxt)  {\textbf{Input}};
                \node[] at (5.1, 5.5) (outputtxt) {\textbf{Output}};

                \draw [shorten >= 2pt, shorten <= 2pt, arrow] (input) -- (output);
            \end{tikzpicture}

            \bigskip

            \begin{tikzpicture}[scale=0.9, every node/.style={scale=0.9}]
                \node[box] at (-0.1, 4.0) (input) {
                    \begin{minipage}{.28\textwidth}
                        \begin{minted}[autogobble, breaklines, fontsize=\scriptsize, frame=none]{javascript}
                            t.inputImmediate({
                                device: 'mouse',
                                isDown: true,
                                x: 50,
                                y: 100
                            });
                        \end{minted}
                    \end{minipage}
                };

                \node[box] at (5.1, 4.0) (output) {
                    \begin{minipage}{.28\textwidth}
                        \begin{minted}[autogobble, breaklines, fontsize=\scriptsize, frame=none]{javascript}
                            sprite.x
                            sprite.y
                            sprite.rotation
                            sprite.sayText
                            sprite.costume
                            variable.value
                        \end{minted}
                    \end{minipage}
                };

                \node[] at (0.0, 5.5) (inputtxt)  {\textbf{Input}};
                \node[] at (5.1, 5.5) (outputtxt) {\textbf{Output}};

                \draw [shorten >= 2pt, shorten <= 2pt, arrow] (input) -- (output);
            \end{tikzpicture}

            \caption{Comparison of Scratch's IO and Whisker's IO}
        \end{center}
    \end{figure}
\end{frame}

\begin{frame}\frametitle{How to test Scratch programs? Automating IO}
    \begin{figure}
        \centering
        \tikzset{>=latex,
                 arrow/.style={-{Latex[length=1.5mm, width=1.5mm]}},
                 label/.style={draw=none, text width=5.3cm, minimum height=0.5cm, text centered},
                   box/.style={draw,      text width=3.2cm, minimum height=0.7cm, text centered, rounded corners},
                     h/.style={fill=blue!10}}

        \begin{tikzpicture}
            \node[box, h] at ( 0.0, 3.0) (gui)           {Scratch GUI};
            \node[box]    at (-2.0, 1.5) (scratchvm)     {Scratch VM};
            \node[box]    at ( 2.2, 1.5) (scratchrender) {Scratch Renderer};
            \node[box, color=gray]   at (-2.0, 0.0) (program)       {Program};
            \node[box, color=gray]   at ( 2.2, 0.0) (htmlcanvas)    {HTML Canvas};

            \foreach \pp/\pf/\pt in {--/gui/scratchvm,
                                     --/gui/scratchrender,
                                     --/scratchvm/scratchrender,
                                     --/scratchvm/program,
                                     --/scratchrender/htmlcanvas}
            \draw[shorten >= 2pt, arrow] (\pf) \pp (\pt);
        \end{tikzpicture}

        \caption{General architecture of Scratch}
    \end{figure}

    \pause

    \begin{itemize}
        \item[$\rightarrow$] replace Scratch's GUI with an implementation that provides an API to access Scratch's IO
    \end{itemize}
\end{frame}

\begin{frame}\frametitle{How to test Scratch programs? Automating IO}
    \begin{figure}[htpb]
        \centering
        \tikzset{>=latex,
                 arrow/.style={-{Latex[length=1.5mm, width=1.5mm]}},
                 label/.style={draw=none, text width=5.3cm, minimum height=0.5cm, text centered},
                   box/.style={draw,      text width=3.2cm, minimum height=0.7cm, text centered, rounded corners},
                     h/.style={fill=blue!10}}

        \begin{tikzpicture}
            \node[box]    at ( 0.00,  1.5) (testcode)      {Test Code};
            \node[box, h] at ( 0.00,  0.0) (whisker)       {Whisker};
            \node[box]    at (-2.00, -1.5) (scratchvm)     {Scratch VM};
            \node[box]    at ( 2.20, -1.5) (scratchrender) {Scratch Renderer};
            \node[box, color=gray]   at (-2.0, -3.0) (program)       {Program};
            \node[box, color=gray]   at ( 2.2, -3.0) (htmlcanvas)    {HTML Canvas};

            \foreach \pp/\pf/\pt in {--/testcode/whisker,
                                     --/whisker/scratchvm,
                                     --/whisker/scratchrender,
                                     --/scratchvm/scratchrender,
                                     --/scratchvm/program,
                                     --/scratchrender/htmlcanvas}
            \draw[shorten >= 2pt, arrow] (\pf) \pp (\pt);
        \end{tikzpicture}

        \caption{General architecture of Whisker}
    \end{figure}
\end{frame}

\begin{frame}[fragile]\frametitle{How to test Scratch programs? Automating IO}
    \begin{figure}
        \tikzset{>=latex,
               arrow/.style={draw, -{Latex[length=1.5mm, width=1.5mm]}},
                 put/.style={draw, minimum height=0.65cm, minimum width=1.75cm, rounded corners, fill=red!20, text width=2.4cm, text centered},
                  vm/.style={draw, minimum height=1.75cm, minimum width=3.0cm, rounded corners, fill=white},
                 gui/.style={draw, minimum height=2.6cm, minimum width=3.5cm, rounded corners, fill=blue!20},
             whisker/.style={draw, minimum height=2.6cm, minimum width=3.5cm, rounded corners, fill=green!20},
                 box/.style={draw,text centered, rounded corners}}

        \hspace{2mm}\begin{tikzpicture}[scale=0.9, every node/.style={scale=0.9}]
            \node[box] at (0.0, 4.0) (input)  {\includegraphics[height=.25\textheight]{mouse-keyboard}};
            \node[box] at (8.1, 4.0) (output) {\includegraphics[height=.25\textheight]{scratch-stage}};

            \node[] at (0.0, 5.5) (inputtxt)  {\textbf{Input}};
            \node[] at (8.1, 5.5) (outputtxt) {\textbf{Output}};

            \begin{scope}[on background layer]
                \node[gui] at (4.0,  4.0)  (gui) {};
                \node[vm]  at (4.0,  3.75) (vm)  {};
            \end{scope}

            \node[put] at (4.0,  3.5) (put)     {\small Program under test};
            \node[]    at (4.0,  4.3) (vmtxt)   {\small Scratch VM};
            \node[]    at (4.0,  4.95) (guitxt) {\textbf{Scratch GUI}};

            \draw [shorten >= 2pt, shorten <= 2pt, arrow] (input) -- (gui);
            \draw [shorten >= 2pt, shorten <= 2pt, arrow] (gui)   -- (output);
        \end{tikzpicture}

        \bigskip

        \begin{tikzpicture}[scale=0.9, every node/.style={scale=0.9}]
            \node[box] at (-0.1, 4.0) (input) {
                \begin{minipage}{.28\textwidth}
                    \begin{minted}[autogobble, breaklines, fontsize=\scriptsize, frame=none]{javascript}
                        t.inputImmediate({
                            device: 'mouse',
                            isDown: true,
                            x: 50,
                            y: 100
                        });
                    \end{minted}
                \end{minipage}
            };

            \node[box] at (8.1, 4.0) (output) {
                \begin{minipage}{.28\textwidth}
                    \begin{minted}[autogobble, breaklines, fontsize=\scriptsize, frame=none]{javascript}
                        sprite.x
                        sprite.y
                        sprite.rotation
                        sprite.sayText
                        sprite.costume
                        variable.value
                    \end{minted}
                \end{minipage}
            };

            \node[] at (0.0, 5.4) (inputtxt)  {\textbf{Input}};
            \node[] at (8.1, 5.4) (outputtxt) {\textbf{Output}};

            \begin{scope}[on background layer]
                \node[whisker] at (4.0,  4.0)  (whisker) {};
                \node[vm]      at (4.0,  3.75) (vm)      {};
            \end{scope}

            \node[put] at (4.0,  3.5)  (put)        {\small Program under test};
            \node[]    at (4.0,  4.3)  (vmtxt)      {\small Scratch VM};
            \node[]    at (4.0,  4.95) (whiskertxt) {\textbf{Whisker}};

            \draw [shorten >= 2pt, shorten <= 2pt, arrow] (input)   -- (whisker);
            \draw [shorten >= 2pt, shorten <= 2pt, arrow] (whisker) -- (output);
        \end{tikzpicture}

        \caption{Comparison of Scratch's IO and Whisker's IO}
    \end{figure}
\end{frame}

\begin{frame}
    \bigcenter{Whisker}
\end{frame}

\begin{frame}\frametitle{Whisker, Whisker's GUI}
    \begin{figure}
        \includegraphics[width=\textwidth]{whisker-gui-big-upscaled}
        \caption{Whisker's GUI}
    \end{figure}
    % Whisker allows us to interact with the Scratch VM through a JavaScript API
\end{frame}

\begin{frame}[fragile]\frametitle{Whisker, Example Test}
    \only<1,3,5,7>{\tikzset{test/.style={opacity=1.0}}}
    \only<2,4,6>{\tikzset{test/.style={opacity=0.3}}}
    \vspace{-\bigskipamount}
    \begin{minipage}{.75\textwidth}\begin{tikzpicture}
        \node[test] (0,0) {
            \begin{minipage}{\textwidth}
                \begin{javascriptcode}
                    const test = async function (t) {
                        await t.runForTime(100);

                        const sprite = t.getSprite('Sprite1');
                        let oldX = sprite.x;

                        await t.runForTime(250);

                        t.assert.ok(oldX === sprite.x);

                        t.inputImmediate({
                            device: 'keyboard',
                            key: 'right arrow',
                            isDown: true
                        });

                        await t.runForTime(250);

                        t.assert.ok(sprite.x > oldX);
                    }
                \end{javascriptcode}
            \end{minipage}
        };
    \end{tikzpicture}\end{minipage}%
    \hspace{-.4\textwidth}%
    \begin{onlyenv}<2>
        \begin{minipage}{.6\textwidth}
            \begin{javascriptcode}
                /* Running the program */

                /* Tests start with the program in a
                 * paused state. The test can then run
                 * (resume) the program on demand. */

                await t.runForTime(500);
                await t.runUntil(() => a > b, 1000);

                t.getRunTimeElapsed();
                t.getTotalTimeElapsed();

                t.greenFlag();
            \end{javascriptcode}
        \end{minipage}
    \end{onlyenv}
    \begin{onlyenv}<4>
        \begin{minipage}{.6\textwidth}
            \begin{javascriptcode}
                /* Accessing sprites and variables */

                /* Accessed through objects that always
                 * have the current sprite attributes /
                 * variable value. */

                t.getSprite('Sprite1');
                t.getSprites(sprite => sprite.x > 100);
                t.getStage();

                sprite.getVariable('my variable');

                sprite.x;
                sprite.rotation;
                variable.value;

                sprite.isTouchingEdge();
            \end{javascriptcode}
        \end{minipage}
    \end{onlyenv}
    \begin{onlyenv}<6>
        \begin{minipage}{.6\textwidth}
            \begin{javascriptcode}
                /* Simulating Inputs */

                t.inputImmediate({
                    device: 'keyboard',
                    key: 'space',
                    isDown: true,
                    duration: 100
                });
                t.addInput(1000, {
                    device: 'mouse',
                    x: 100,
                    y: 200,
                    isDown: true
                });

                t.getMousePos();
                t.isKeyDown('space');
            \end{javascriptcode}
        \end{minipage}
    \end{onlyenv}
\end{frame}

\begin{frame}[fragile]\frametitle{Whisker, Callbacks}
    \begin{itemize}
        \item Callbacks get executed every time a frame is rendered
        \item Make it possible to \textcolor{upfim}{track} and \textcolor{upfim}{react} to what the user sees
    \end{itemize}

    \begin{javascriptcode}
        t.addCallback(() => someList.push(sprite.x));

        const callback = t.addCallback(() => {
            if (sprite.visible) {
                t.inputImmediate({ device: 'mouse', isDown: true });
            } else {
                t.inputImmediate({ device: 'mouse', isDown: false });
            }
        });

        callback.disable();
        callback.enable();
        callback.isActive();
    \end{javascriptcode}
\end{frame}

\begin{frame}[fragile]\frametitle{Whisker, Constraints}
    \begin{itemize}
        \item Constraints \textcolor{upfim}{define conditions} that must hold for the program
        \item Like callbacks, constraints are checked (executed) every time a frame is rendered
    \end{itemize}

    \begin{javascriptcode}
        t.onConstraintFailure('fail');
        t.onConstraintFailure('nothing');

        const constraint = t.addConstraint(() => {
            t.assert.ok(sprite.visible === true, 'Sprite must be visible.');
        });

        constraint.disable();
        constraint.enable();
        constraint.isActive();
    \end{javascriptcode}
\end{frame}

\begin{frame}[fragile]\frametitle{Whisker, Automated Input Generation}
    \begin{itemize}
        \item At a constant frequency, performs a random input from a pool
        \item Whisker can detect what inputs the program can react to
    \end{itemize}

    \begin{javascriptcode}
        t.setRandomInputInterval(150);

        t.registerRandomInputs([
            { device: 'keyboard', key: 'left arrow', duration: [50, 100] },
            { device: 'keyboard', key: 'right arrow', duration: [50, 100] },
            { device: 'mouse', x: [-50, 50], y: [-100, 100], weight: 0.5 }
        ]);

        t.detectRandomInputs({ duration: [50, 100] });
    \end{javascriptcode}
\end{frame}

\newcommand{\tablebox}[1]{
    \begin{tikzpicture}
         \node[draw, text width=7.5cm, minimum height=0.6cm, rounded corners] {\footnotesize #1};
    \end{tikzpicture}
}

% \begin{frame}\frametitle{Whisker, Detecting Inputs}
%     \scalebox{0.75}{
%         \begin{tabular}{m{5.25cm}m{9.0cm}}
%             \large Scratch Block(s) & \large Resulting Input(s) \\
%             \vspace{3.5mm}\includegraphics[scale=1.5]{scratch-input-blocks-1}\vspace{2mm} & \tablebox{(1) Press the respective keyboard key}                \\
%             \vspace{3.5mm}\includegraphics[scale=1.5]{scratch-input-blocks-4}\vspace{2mm} & \tablebox{(1) Move the cursor near / onto the respective sprite \\
%                                                                                                       (2) Move the cursor to a random position}             \\
%             \vspace{3.5mm}\includegraphics[scale=1.5]{scratch-input-blocks-7}\vspace{2mm} & \tablebox{(1) Answer with a randomly generated string}          \\
%             \vspace{3.5mm}\includegraphics[scale=1.5]{scratch-input-blocks-8}\vspace{2mm} & \tablebox{(1) Answer with the compared string constant}         \\
%             \huge ...                                                                     & \huge ...                                                       \\
%         \end{tabular}
%     }
% \end{frame}
