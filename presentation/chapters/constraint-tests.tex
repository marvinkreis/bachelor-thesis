\begin{frame}
    \bigcenter{Property-based Testing with Whisker}
\end{frame}

\begin{frame}[fragile]\frametitle{Property-based Testing, QuickCheck}
    Example: QuickCheck

    \medskip

    \begin{itemize}
        \item Define properties
        \item Feed program with (random) input
        \item Check if the program complies to the defined properties
    \end{itemize}

    \pause
    \medskip

    \begin{figure}
        \begin{minted}[autogobble, breaklines, fontsize=\scriptsize, frame=single]{haskell}
            prop_RevApp xs ys =
                reverse (xs++ys) == reverse xs++reverse ys
        \end{minted}

        \begin{minted}[autogobble, breaklines, fontsize=\scriptsize, frame=single]{bash}
            Main> quickCheck prop_RevApp
            OK: passed 100 tests.
        \end{minted}

        \begin{minted}[autogobble, breaklines, fontsize=\scriptsize, frame=single]{bash}
            Main> quickCheck prop_RevApp
            Falsifiable, after 1 tests:
            [2]
            [-2,1]
        \end{minted}

        \caption{QuickCheck Usage Example (from~\cite{quickcheck})}
    \end{figure}
\end{frame}

\begin{frame}\frametitle{Property-based Testing, Whisker}
    Property-based testing can be performed on Scratch programs in a different way,
    by checking properties in the background, while the program is running.

    \pause
    \bigskip

    \begin{itemize}
        \item \textcolor{upfim}{Callbacks:} Observe the program's output
        \item \textcolor{upfim}{Constraints}: Define properties
        \item (\textcolor{upfim}{Automated Input Generation}: Control the program)
    \end{itemize}
\end{frame}

% \begin{frame}
%     \bigcenter{Why property-based testing for Scratch?}
% \end{frame}
%
% \begin{frame}\frametitle{Why property-based testing for Scratch?}
%     \begin{itemize}
%         \item Many conditions can be checked in a single test case
%             \begin{itemize}
%                 \item less test code
%                 \item less execution time
%             \end{itemize}
%
%         \bigskip
%
%         \item Makes tests independent of the source of input
%             \begin{itemize}
%                 \item generated input
%                 \item recorded input
%             \end{itemize}
%     \end{itemize}
% \end{frame}

% TODO: resets
