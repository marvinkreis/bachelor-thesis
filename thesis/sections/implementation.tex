\section{Implementation}

\subsection{Design Considerations}
- Lack of traditional IO means makes it hard to extract information about the current program state
- Interactive nature of Scratch projects makes it difficult to simulate user input
- Black Box approach means we only test what the user sees

\subsection{Implementation}
- Implemented in JavaScript as Scratch 3.0 is implemented in JavaScript

\begin{figure}
    \tikzset{
        >=latex,
        label/.style={draw=none, text width=5.3cm, minimum height=0.5cm, text centered},
          box/.style={draw,      text width=2.5cm, minimum height=0.7cm, text centered},
         mute/.style={color=gray}
    }
    \centering
    \begin{tikzpicture}
        \node[box,mute] at ( 0.0,  3.0) (testfw)        {Test Framework};
        \node[box]      at ( 0.0,  1.5) (testdriver)    {Test Driver};
        \node[label]    at ( 0.0,  0.0) (vmwrapper)     {VM Wrapper};
        \node[box]      at (-1.4, -0.7) (sprites)       {Sprites};
        \node[box]      at (-1.4, -1.6) (inputs)        {Inputs};
        \node[box]      at ( 1.5, -0.7) (callbacks)     {Callbacks};
        \node[box]      at ( 1.5, -1.6) (constraints)   {Constraints};
        \node[box,mute] at (-2.0, -3.2) (scratchvm)     {Scratch VM};
        \node[box,mute] at ( 2.2, -3.2) (scratchrender) {Renderer};

        \node[draw, fit=(vmwrapper)(sprites)(inputs)(callbacks)(constraints)] (container) {};

        \foreach \pp/\pf/\pt in {--/testfw/testdriver,
                                 --/testdriver/container,
                                 --/container/scratchvm,
                                 --/container/scratchrender,
                                 --/scratchvm/scratchrender}
        \draw[shorten >=\pgflinewidth, ->] (\pf) \pp (\pt);
    \end{tikzpicture}
    \caption{Components of Whisker}
    \label{fig:components_of_whisker}
\end{figure}

\subsection{Running Tests}
- Whisker comes with an optional testing framework
- Include a sample test report?
