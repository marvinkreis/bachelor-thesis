% Document layout
\documentclass[12pt,a4paper,twoside]{report}
\usepackage[margin=3.cm]{geometry}
\raggedbottom

% Standard packages
\usepackage[utf8]{inputenc}
\usepackage[bookmarks,hidelinks]{hyperref}
\usepackage{graphicx}
\graphicspath{{media/}}

% List formatting
\usepackage{enumerate}
%\usepackage{enumitem}

% Math symbols
\usepackage{amsfonts}

% URLs for references
\usepackage{url}

% Title page
\usepackage{titling}
\renewcommand*{\maketitle}{
\begin{titlepage}

    \centering

    \begin{minipage}[c]{0.45\textwidth}
        {\includegraphics[width=\textwidth]{media/uni-logo}\par}
    \end{minipage}
    \begin{minipage}[c]{0.05\textwidth}
        {~}
    \end{minipage}
    \begin{minipage}[c]{0.4\textwidth}
        {\flushleft\Large University of Passau \unskip\strut\par}
        {\flushleft\large Chair of Software Engineering II \unskip\strut\par}
    \end{minipage}

    \vspace{2cm}

    {\huge\bfseries \thetitle \unskip\strut\par}

    \vspace{2cm}

    {\Large \theauthor \unskip\strut\par}
    {\Large 77280 \unskip\strut\par}

    \vspace{2cm}

    {\Large Prof. Dr. Gordon Fraser \unskip\strut\par}

    \vfill

    {\large In partial fulfillment of the requirements for the degree of \unskip\strut\par}
    {\large B.Sc. Computer Science \unskip\strut\par}

    \vfill

    {\large \thedate \par}

\end{titlepage}
}


% Tikz diagrams
\usepackage{tikz}
\usetikzlibrary{fit, backgrounds, shapes, arrows, arrows.meta, chains}

% Tables
\usepackage{booktabs}
\usepackage{multirow}
\usepackage{array}
\usepackage{makecell}
\renewcommand{\cellalign}{l}

% Checkmarks and crosses
\usepackage{pifont}
\newcommand{\cmark}{\ding{51}}
\newcommand{\xmark}{\ding{53}}
\newcommand{\smark}{\ding{73}}

% Source code highlighting
% outputdir workaround because minted ignores the '-output-directory' flag.
\usepackage[outputdir=/tmp/latexrun]{minted}
\newminted{javascript}{
    autogobble,
    breaklines,
    linenos,
    fontsize=\footnotesize,
    framesep=2mm,
    frame=lines}
\newmintinline{javascript}{}

% Figures
\usepackage[justification=centering, font=small]{caption}
\usepackage{subcaption}
\usepackage[section]{placeins}

% Line spacing
\renewcommand{\baselinestretch}{1.1}

% Page style
\usepackage{fancyhdr}
\fancypagestyle{plain}{
    \fancyhead{}
    \fancyhead[LE,RO]{\thepage}
    \fancyhead[LO]{\leftmark}
    \fancyhead[RE]{\rightmark}
    \fancyfoot{}}
\pagestyle{plain}
\setlength{\headheight}{20pt}

% Space between paragraphs
\newcommand{\parspace}{\bigskip}

% Notes
\usepackage{marginnote}
\newcommand{\mnote}[1]{\marginnote{\parbox{2cm}{\raggedright\scriptsize #1}}}

% Tables and listings all in the same list
\def\table{\def\figurename{Table}\figure}
\let\endtable\endfigure
\def\listing{\def\figurename{Listing}\figure}
\let\endlisting\endfigure
\renewcommand\listfigurename{List of Figures, Tables and Listings}

% Calculate sizes
\usepackage{calc}

% Math
\usepackage{amsmath}

% Information
\title{Whisker: Automated Testing of Scratch Programs}
\author{Marvin Lorenz Kreis}
\date{DATE TBD}

\begin{document}

% https://sokogskriv.no/en/writing/structure-and-argumentation/structuring-a-thesis/
% https://sokogskriv.no/en/writing/language-and-style/organising-your-writing/

% Write intro + background parallel to the main part and only gather some data before.

% RQs: Can Scratch programs be automatically assessed?
%      Can this be done without simulating user input / just with constraints?

% Title
\maketitle

% \setlength{\parskip}{.5\smallskipamount}

% Table of Contents
\tableofcontents
\clearpage

% \let\oldclearpage=\clearpage
% \renewcommand{\clearpage}{}
\listoffigures
%\listoflistings
% \let\clearpage=\oldclearpage

% \setlength{\parskip}{\bigskipamount}

% Abstract
% REMEMBER: Write the thesis from the view of the reader. How would I like to READ the thesis?
% WHY -> WHAT -> HOW structure

\begin{abstract}
    - Scratch is very common as an introduction to programming
    - Workshops and courses are held with Scratch
    - Many people pick up Scratch on their own and follow tutorials

    - Assessment of Scratch programs is still an open problem
        - Teachers might want to check their student's programs to grade them or to compile statistics
        - Learners might want to check their own solutions and get feedback

    - Automatic assessment of Scratch programs is needed
        - basic static analysis possible
        - runtime testing is not really possible yet
        - Scratch does not have normal input / output like traditional programs, but is based on multimedia
        - programs need to be run manually to assess their functionality
        - takes a lot of effort
        - Whisker can support educators and learners in the assessment

    - Whisker allows GUI testing for Scratch
        - Allows to write test cases in JavaScript that executed the program, simulate user input and check sprite properties

    - Whisker was tested on a collection of Scratch programs written by students
    - Evaluation shows ...
\end{abstract}

%     Explain what this thesis is about, and give a quick summary.
%     Give the reader a good idea about what's going on.

\chapter{Introduction}

\section{Motivation}

Introductory computer science (CS) courses often use educational programming languages to teach the principles of computer programming.
These languages are designed to be easily understandable and engaging for programming novices.
In order to accomplish this, they often feature visual and auditory projects instead of textual input and output (IO).
One of the best known and most used languages for this purpose is MIT's Scratch~\cite{scratch, scratchproject}.
Usages of Scratch can be found from primary school classes all the way up to introductory CS courses in universities~\cite{itch}.
Scratch features a block-based programming environment, which lets users build interactive, multimedia programs with little effort.
\parspace

But Scratch's multimedia focus can also be its downfall.
Scratch is entirely developed and executed inside of a graphical user interface (GUI), and programs usually require manual interaction with keyboard and mouse to run.
Because of this, grading student assignments is troublesome and particularly slow.
Assessment of Scratch programs involves opening, running and interacting with every program individually.
Some courses, that use Scratch, are attended by more than 200 students~\cite{itch}.
Therefore, manual assessment of student solutions becomes too time consuming to be feasible, and automated assessment is needed to grade assignments.
For traditional programming languages with text-based IO, functional testing can be deployed in order to evaluate assignments in a less time consuming, less error-prone, automated way.
Programs are given some pre-defined input, then their according output is analyzed and checked for correctness.
But therein lies Scratch's problem.
Scratch does not have textual input and output mechanisms like most programming languages.
Therefore, automatically testing Scratch programs is not a trivial task and still poses an open problem.
\parspace

Besides automated grading, dynamic testing for Scratch programs can also be useful for students.
For one thing, teachers can provide their students with test suites, so they can run tests on their implementations themselves.
By doing this, students can receive valuable feedback about their programs.
They can possibly identify and fix errors in their solutions before submitting them.
In some cases, maybe even a form of test driven development (TDD) can be adopted.
With TDD, students write their program bit by bit in order to incrementally satisfy an existing test suite.
Likewise, dynamic testing is also helpful for self-study.
Online tutorials for Scratch could include a test suite for learners to test their programs with.
Learners could verify their solution, or receive feedback on possible errors in their solution.

\section{Contributions}

This thesis introduces a new approach towards dynamic testing of Scratch programs by automating Scratch's IO.
The testing procedure involves a program, which simulates user input (e.g.\ mouse movement or key presses)
and allows access to Scratch's visual output by providing information about the program's sprites and variables.
We present a way to perform property-based testing \cite{quickcheck} for Scratch,
which opens up the possibility to use arbitrary sources of input for testing,
including automatically generated input.
\parspace

As the main contribution of this work, we introduce Whisker, an implementation of the aforementioned program.
Whisker provides an interface to control the Scratch virtual machine through JavaScript methods.
Whisker's source code is available at \url{https://github.com/marvinkreis/whisker}.
With this implementation, we also explore the possibility of automated test input generation through a combination of random input and simple static analysis on the Scratch project.
\parspace

We evaluate Whisker with realistic Scratch programs from different educational workshops and courses.
The results provide the following insights:

\begin{enumerate}
    \item[(1)] Test results can be accurate enough to aid in grading Scratch assignments.
        Tests are able closely match the results of manual assessment and show fairly consistent results over multiple test runs.
        We measured an average Pearson's correlation coefficient of $r = 0.882$ for the correlation
        between our test results and independent manual scores.
        And we observed a percentage of $4.52\%$ of test-project combinations showing inconsistent test outcomes over ten test executions.
    \item[(2)] A combination of random input generation and simple static analysis can be used to generate input, which covers a big portion of
        typical Scratch programs' functionality.
        We were able to achieve an average of $95.25\%$ statement coverage on a variety of projects with automatically generated input,
        while running the projects without inputs only resulted in $47.14\%$ statement coverage.
    \item[(3)] The testing process does not interfere with the execution of the programs under test.
        Specifically, additional computations for testing do not slow down the execution of tested programs.
\end{enumerate}

\section{Outline}

In Chapter~\ref{cha:background}, we first provide a small overview over the Scratch programming language (Section~\ref{sec:scratch}).
Then we examine previous approaches towards automated testing of Scratch programs (Section~\ref{sec:previous_testing_approaches})
and explain some general challenges, which have to be overcome in order to perform automated testing for Scratch (Section~\ref{sec:challenges_of_testing_scratch_programs}).%
\parspace

We explain our approach towards testing Scratch programs (Section~\ref{sec:general_appraoch}),
and how we realized this approach with Whisker, in Chapter~\ref{cha:appraoch}.
We describe the environment, in which tests are executed (Section~\ref{sec:testing_environment}),
how tests are written, and what functionality Whisker provides for testing (Section~\ref{sec:public_interface}).
Afterwards, we explain various challenges of this testing approach (Section~\ref{sec:appraoch_challenges}).
\parspace

In Chapter~\ref{cha:using_constraints_for_flexible_test_inputs}, we describe how Whisker can be used to perform property-based testing.
We explain the general idea of this approach (Section~\ref{sec:input_independent_constraint_only_tests})
and the testing procedure we use to achieve this (Section~\ref{sec:constraint_testing_procedure}).
\parspace

Furthermore, we describe the implementation of Whisker in Chapter~\ref{cha:implementation}.
After we explain the implementation environment (Section~\ref{sec:implementation_environment})
and the general architecture (Section~\ref{sec:general_architecture}) of Whisker,
we explain how the controlled execution of Scratch programs is implemented (Section~\ref{sec:scratch_program_execution_and_the_step_loop}).
The following sections then each describe the implementation of one of Whisker's features.
Finally, we explain how Whisker measures statement coverage (Section~\ref{sec:coverage_measurement}).
\parspace

In Chapter~\ref{cha:evaluation}, we perform an empirical evaluation of Whisker.
In the beginning, we list the research questions and give an overview of the experiments we conducted.
Afterwards, we describe the projects and test suites we use for the evaluation (Section~\ref{sec:experimental_setup}),
Then, in each of the following sections,
we explain the experiments and the indicators we use to answer each research question and describe our results.
Firstly, we analyze the accuracy of the test results from our test suites (Section~\ref{sec:rq1}) and the test suites' flakiness (Section~\ref{sec:rq2}),
then we evaluate our algorithm for generating automated test input (Section~\ref{sec:rq3}).
Finally we examine if programs under tests are somehow influenced by the testing process (Section~\ref{sec:rq4}).
Afterwards, we discuss the results (Section~\ref{sec:discussion}),
and list possible threats to validity (Section~\ref{sec:threats_to_validity}).
\parspace

Finally, in Chapter~\ref{cha:future_work} we describe how Whisker could be extended in the future,
then we conclude in Chapter~\ref{cha:conclusion}.


%     Overview of the main points.
%     Discuss the research questions.
%     Define the scope of the thesis.
%     Schematic outline for the rest of the thesis.
%
%     Awaken the reader's interest.
%     Should connect with the conclusions, so review / rewrite it in the end.

% REMEMBER: Write the thesis from the view of the reader. How would I like to READ the thesis?

\chapter{Background}

This chapter will give some background about how Scratch works and why testing Scratch programs is a difficult task.
It will also highlight some related work, which has tackled to problem of testing Scratch programs.

\section{Scratch}

\begin{wrapfigure}{r}{0.35\textwidth + 5mm}
    \centering
    \vspace{-3mm}
    \includegraphics[width=0.35\textwidth]{scratch-stage}
    \caption{A catching game implemented in Scratch}
    \label{fig:a_catching_game_implemented_in_scratch}
\end{wrapfigure}
- Block based programming language developed by the MIT Media Lab~\cite{scratch}
- Aimed towards novices and children
- Block-based code eliminates the possibility of syntax errors
- Code creates interactive two-dimensional animations on a stage
- Possible to control the program with keyboard and mouse input, often creating game-like programs
    - Extensions offer more possibilities
- Heavily focused on multimedia, graphics and audio can easily be integrated into Scratch projects
\parspace

\begin{wrapfigure}{r}{0.45\textwidth + 5mm}
    \centering
    \vspace{-3mm}
    \includegraphics[width=0.45\textwidth]{scratch-code}
    \caption{Scratch blocks}
    \label{fig:scratch_blocks}
\end{wrapfigure}
- Code is made up by \textit{blocks}
- Multiple blocks combined together with a \textit{hat} make up a \textit{script}
- Scripts are called through their hats, they can be called by
    - input events, e.g. key press, mouse click
    - other scripts and sprites through by broadcasting messages
    - green flag, the program is started through the green flag

- All currently running scripts conceptually run in parallel
- Scratch VM sequentializes $\rightarrow$ no race conditions on the language level
\parspace

\begin{wrapfigure}{r}{0.45\textwidth + 5mm}
    \centering
    \vspace{-4mm}
    \includegraphics[width=0.45\textwidth]{scratch-sprites}
    \caption{The sprite menu}
    \label{fig:the_sprite_menu}
\end{wrapfigure}
- Visual objects are represented as sprites
- The code of the program is split between the sprites
- Each sprite and the stage have their own code which manipulates them and can interact with the code of other sprites
- Sprites have variables, which they can access
- Every sprite can access the stage's variables
\parspace

% \begin{wrapfigure}{r}{0.25\textwidth + 5mm}
%     \centering
%     \vspace{-3mm}
%     \includegraphics[width=0.25\textwidth]{scratch-code-clone}
%     \caption{Creating clones}
%     \label{fig:creating_clones}
% \end{wrapfigure}
% - To create a second instance of a sprite, a sprite can clone itself
% - The new sprite then runs scripts that have a ''when I start as a clone'' hat
% - The new clone accesses the same variables as the original sprite, variables are not copied
% \parspace
%
% Lorem ipsum dolor sit amet, consectetur adipiscing elit. Etiam cursus neque at magna vehicula, id lobortis massa tempus. Ut vitae orci euismod, posuere purus at, fringilla justo. Suspendisse in sodales neque. Mauris odio enim, varius sed porttitor in, ullamcorper nec eros. Aenean placerat arcu non leo imperdiet sodales. Nullam vestibulum justo nibh. Duis ac nibh congue, molestie sapien in, condimentum turpis. Donec in nisl vitae sem vehicula vehicula ac non velit. Integer at laoreet sapien. Lorem ipsum dolor sit amet, consectetur adipiscing elit. Aliquam suscipit arcu condimentum rhoncus auctor. Donec porta augue vitae quam malesuada posuere.

\section{Previous Testing Approaches}

- Since the multimedia nature of Scratch programs makes testing Scratch programs difficult,
  automatic assessment of Scratch programs is still an open problem
- At least two other projects have tackled the task of automatically assessing Scratch projects.

=== Hairball
- Hairball \cite{hairball} allows static analysis of scratch programs
- Written in Python
- Takes the source file of a scratch project and performs static analysis
- Has a plugin-based architecture
    - Comes with pre-defined plugins, for example to detect dead code or wrong synchronization between broadcast blocks and their receivers
    - allows to define own plugins, which iterate over the scratch blocks to analyze them
- Possible use cases:
    - check if students use a new construct, which the exercise focuses on
    - detect code smells
    - measure the complexity of a program $\rightarrow$ see Dr.Scratch
- Since it only allows static analysis, not suitable to check the functionality of a program
- Already used in DrScratch \cite{drscratch} to automatically determine the complexity of Scratch programs.

=== ITCH
- ITCH (Individual Testing of Computer Homework for Scratch Assignments)~\cite{itch}
- set of Python scripts
- allows simple textual input-output testing with pre-defined values using "ask" and "say" blocks
- Works by replacing "ask" and "say" blocks in the project
    - "ask" and "say" blocks are used in Scratch to ask the user for textual input and to display textual output on the screen
      (use cartoon speech bubbles)
    - "ask" blocks are replaced by configured test input
    - "say" blocks save the value into a variable instead of displaying it
    $\rightarrow$ the value can then be retrieved by saving the project and reading the value from the saved project
    - TODO picture of say and ask blocks
- This, however, only allows to test projects with a small subset of Scratch functionality
- Only useful for very specific tasks that don't have much to do with Scratch's strengths
- Not very useful in general for Scratch because it focuses on multimedia and creativity
- Other languages are better suited for this task: BlockPy~\cite{blockpy}
    - Has similar operators
    - Provides a way to convert the block constructs to Python code, which can be automatically tested easily

\section{Challenges of Testing Scratch Programs}
- Runtime testing of Scratch programs has some caveats
- This section explains some critical details that make testing of Scratch programs difficult

=== Parallel scripts
- Individual testing of program parts problematic
    - Scratch runs multiple scripts in parallel
        - Scripts often depend on one another
    - Not very useful to test single program parts
    - Possible to define multiple independent scripts that each activate with a different button press
        - but becomes convoluted
        - but defeats to purpose of Scratch to create a interactive environment
$\rightarrow$ Use a complete black box approach
    - Scratch programs are controlled by simulating user input instead of manually calling certain scripts
    - The only information, which is checked, is what appears on screen
        - properties of sprites and variables

=== Lack of IO, multimedia nature
- Scratch has a Lack of traditional IO mechanisms
    - makes it hard to extract information about the current program state
    - makes it hard to control the execution of a Scratch program
    - no traditional input output testing possible without limiting the project to a small subset of Scratch's functionality
$\rightarrow$ Give test cases an interface that can be used to interact with the Scratch project
    - interacts directly with the Scratch virtual machine
    - a way to get information about sprites and variables
    - a way to simulate user input

=== Many tested properties will depend on time
- A tested property might depend on a previous value
    - e.g. check if a sprite is moving right
    - sprites save the values from the previous execution step
- Some tested properties should hold for a time or for the whole execution
    - Provide a way to define constraints that must always hold

\section{Black Box Testing}

- Black box approach:
    - Block box testing bases tests on the specification of the tested program
    - Tests are written without knowledge of the internals of a program
    - Program is seen as a ''black box'' that takes some input and produces some output.
    - Program specification is in the case of Scratch programs most likely a task description for some course or tutorial

    - Input for the black box is user interaction through mouse and keyboard
    - Interact only how a user can interact with the program
        - Since Scratch can only be controlled through user interaction and has no API to call blocks or scripts,
          it only makes sense to control the program this way in the test
        - No information about the internals of the program other than sprite and variable names
            - Sprite names should be included in the specification so the test can more easily identify sprites
            - Giving no or little information about the implementation makes sense for a task description
    - Output of the black box are changes on Scratch's stage
    - Test only what the user sees
        - Only information about sprites and variables (both can be shown on screen)
        - Sprite positions, movement, looks, etc.

\section{Automatic Test Generation}%
\label{sec:automatic_test_generation}




\chapter{Testing with Whisker}
\label{cha:appraoch}

\section{General Approach}
\label{sec:general_appraoch}

In this work, we propose a way to perform dynamic testing on Scratch programs for Scratch 3.0.
The main goal of this approach is to be able to automatically assess student's solutions to Scratch assignments.
In order to do so, this approach makes use of an automation utility, which allows test code to
interact with running Scratch programs through Scratch's IO mechanisms.
% To enable dynamic testing for Scratch, we propose a black-bock approach with a testing utility, which allows test code to interact with a running Scratch program.
\parspace

Because Scratch's parallel scripts, as well as its lack of code separation, would make testing individual parts of programs difficult,
we instead chose to test full programs on the level of system tests, and only focus on the program's input and output.
This raises the question of how to access Scratch's IO.
Since Scratch's input usually consists of mouse and keyboard input, and its output consists of visual animations and sound,
the IO is not easily accessible in a programmable way.
To overcome this challenge, we developed an automation utility called Whisker, which acts as a wrapper around Scratch.
It interacts with Scratch's virtual machine in order to automate its input and output.
Whisker offers a programmable interface for Scratch, which allows tests to simulate inputs and access information about sprites and variables.
This makes automated testing for Scratch possible.
% Selenium~\cite{selenium}, a popular tool for automated testing of web sites,
% uses a similar approach to automate web browsers.
Figure~\ref{fig:comparison_of_io_mechanisms} illustrates the difference between Scratch's IO mechanisms and Whisker's automated input and output.

\begin{figure}[htpb]
    \centering

    \begin{subfigure}[b]{\textwidth}
        \centering
        \tikzset{>=latex,
               arrow/.style={draw, -{Latex[length=1.5mm, width=1.5mm]}},
                 put/.style={draw, minimum height=1.7cm, minimum width=3.5cm, rounded corners, fill=red!20, text width=2.5cm, text centered},
                  vm/.style={draw, minimum height=3.0cm, minimum width=6.0cm, rounded corners, fill=white},
                 gui/.style={draw, minimum height=4.2cm, minimum width=7.0cm, rounded corners, fill=blue!20},
                 box/.style={draw, minimum height=4.2cm, minimum width=4.0cm, rounded corners, text width=3.5cm},
              boxtxt/.style={minimum width=4.0cm, rounded corners, text width=3.5cm}}

         \begin{tikzpicture}[scale=0.8, every node/.style={scale=0.8}]
            \begin{scope}[on background layer]
                \node[gui] at (0.0,  0.4) (gui)     {};
                \node[vm]  at (0.0,  0.0) (vm)      {};
            \end{scope}

            \node[put]     at (0.0, -0.4) (put)     {\textbf{Program under test}};
            \node[]        at (0.0,  1.0) (vmtxt)   {\textbf{Scratch Virtual Machine}};
            \node[]        at (0.0,  2.0) (guitxt)  {\large \textbf{Scratch GUI}};
            \node[box, left=of gui]       (input)   {};
            \node[box, right=of gui]      (output)  {};
            \node[boxtxt, below right] at ([yshift=-2mm] input.north west)  (inputtxt)
                {\centering {\large \textbf{Input}}\\[.5\baselineskip]Key presses, mouse movement, mouse clicks, etc.};
            \node[boxtxt, below right] at ([yshift=-2mm] output.north west) (outputtxt)
                {\centering {\large \textbf{Output}}\\[.5\baselineskip]Visual animations, audio, etc.};

            \path [arrow] (input) -- (gui);
            \path [arrow] (gui)   -- (output);
        \end{tikzpicture}
        \caption{Input and output of the Scratch GUI}
    \end{subfigure}

    \bigskip

    \begin{subfigure}[b]{\textwidth}
        \centering
        \tikzset{>=latex,
               arrow/.style={draw, -{Latex[length=1.5mm, width=1.5mm]}},
                 put/.style={draw, minimum height=1.7cm, minimum width=3.5cm, rounded corners, fill=red!20, text width=2.5cm, text centered},
                  vm/.style={draw, minimum height=3.0cm, minimum width=6.0cm, rounded corners, fill=white},
             whisker/.style={draw, minimum height=4.2cm, minimum width=7.0cm, rounded corners, fill=green!20},
                 box/.style={draw, minimum height=4.2cm, minimum width=4.0cm, rounded corners, text width=3.5cm},
              boxtxt/.style={minimum width=4.0cm, rounded corners, text width=3.5cm}}

         \begin{tikzpicture}[scale=0.8, every node/.style={scale=0.8}]
            \begin{scope}[on background layer]
                \node[whisker] at (0.0,  0.4) (whisker) {};
                \node[vm]      at (0.0,  0.0) (vm)      {};
            \end{scope}

            \node[put]         at (0.0, -0.4) (put)     {\textbf{Program under test}};
            \node[]            at (0.0,  1.0) (vmtxt)   {\textbf{Scratch Virtual Machine}};
            \node[]            at (0.0,  2.0) (guitxt)  {\large \textbf{Scratch GUI}};
            \node[box, left=of whisker]       (input)   {};
            \node[box, right=of whisker]      (output)  {};
            \node[boxtxt, below right] at ([yshift=-2mm] input.north west)  (inputtxt)
                {\centering {\large \textbf{Input}}\\[.5\baselineskip]Simulated user inputs through a programmable interface};
            \node[boxtxt, below right] at ([yshift=-2mm] output.north west) (outputtxt)
                {\centering {\large \textbf{Output}}\\[.5\baselineskip]Interface to access information about sprites and variables};

            \path [arrow] (input)   -- (whisker);
            \path [arrow] (whisker) -- (output);
        \end{tikzpicture}
        \caption{Input and output of Whisker}
    \end{subfigure}

    \caption{Comparison of IO mechanisms between the Scratch GUI and Whisker}
    \label{fig:comparison_of_io_mechanisms}
\end{figure}

\section{Testing Environment}
\label{sec:testing_environment}

Whisker is, like Scratch 3.0, implemented in JavaScript (JS).
Hence, test code is also written in JavaScript.
Whisker can theoretically be used with any JS testing framework,
but for compatibility reasons, we developed a rudimentary testing framework to go along with Whisker.
\parspace

Currently, Whisker is only available in its own web GUI, which can be seen in Figure~\ref{fig:whisker_gui}.
The web page displays Scratch's stage, a table of loaded tests, and a test report in TAP13~\cite{tap} format.
It supports batch testing more than one program with the same test suite,
but it doesn't support parallel test execution.
In the future, we plan on implementing a standalone Electron~\cite{electron} application for Whisker.
This would facilitate batch testing many programs,
and would also make it possible to test programs in parallel.
Unfortunately it is not possible to run tests in a headless environment,
because Scratch depends on a HTML canvas to render its output to.
Without a renderer, some of Scratch's blocks don't work properly.

\begin{figure}[htpb]
    \centering
    \includegraphics[width=0.85\textwidth]{whisker-gui-big}
    \caption{Screenshot of Whisker's web GUI}
    \label{fig:whisker_gui}
\end{figure}

\section{Public Interface}
\label{sec:public_interface}

Automating Scratch allows us to write tests for Scratch in a unit-test-like fashion.
Whisker loads the program before each test case in order to create the same initial state for every test case,
instead of going on where the last execution left off, like Scratch's GUI does it.
It also starts the program with the green flag when the test begins.
\parspace

Tests use a test driver object to automate Scratch through Whisker.
Whisker's own testing framework automatically passes the test driver object to its tests as an argument,
but tests written for other testing frameworks may have to acquire the test driver in their test code.
Whisker offers an interface to create and configure the test driver object for this purpose.
Listing~\ref{fig:examples_of_how_to_acquire_the_test_driver} shows code examples for both possibilities.
\parspace

\begin{listing}[htpb]
    \centering
    \begin{subfigure}[b]{.35\textwidth}
        \begin{minted}[autogobble, breaklines, linenos, fontsize=\scriptsize, framesep=2mm, frame=lines]{javascript}
            async function test (t) {
                ...
            }
        \end{minted}
        \vspace{-\bigskipamount}
        \caption{Getting the test driver passed as a parameter}
    \end{subfigure}
    \hspace{.05\textwidth}
    \begin{subfigure}[b]{.50\textwidth}
        \begin{minted}[autogobble, breaklines, linenos, fontsize=\scriptsize, framesep=2mm, frame=lines]{javascript}
            async function test () {
                const whisker = new WhiskerUtil(vm, project);
                whisker.prepare();
                const t = whisker.getTestDriver();
                whisker.start();
                ...
            }
        \end{minted}
        \vspace{-\bigskipamount}
        \caption{Manually acquiring the test driver through Whisker's interface}
    \end{subfigure}
    \caption{Acquiring the test driver}
    \label{fig:examples_of_how_to_acquire_the_test_driver}
\end{listing}

The following list will give an overview of Whisker's basic functions.
Because Whisker is still in early development, some of its methods may undergo changes later.
The test driver object will be denoted as $t$ in example code snippets.

\begin{itemize}
    \item \textbf{Control the program execution.}
        Tests are able to control when and for how long the program under test is run.
        In the beginning of the test, the program starts in a paused state.
        The program can then be run (resumed) for a certain time, or until a condition is met.
        Since the Scratch program execution is asynchronous, we use JavaScript's Promise API for this purpose.
        In order to simply execute the program and wait until the execution is done, the test can be declared as an \texttt{async function} and use the \texttt{await} keyword.

        The green flag can also be pressed again in order to restart the program.
        Additionally, tests can query the currently elapsed time,
        either from the start of the current program execution,
        or from the start of the test case.
        \begin{javascriptcode}
            /* Run the program for one second (1000 ms). */
            await t.runForTime(1000);
            await t.runForSteps(30);

            /* Run the program until a condition is met, with a timeout after one
             * second. The condition is given by a function that returns a boolean. */
            await t.runUntil(() => a > b, 1000);

            /* Get the total time elapsed in the current / last run. */
            t.getRunTimeElapsed();

            /* Get the total time elapsed in this test case. */
            t.getTotalTimeElapsed();

            /* Press the green flag again. */
            t.greenFlag();
        \end{javascriptcode}
    \item \textbf{Simulate Inputs.}
        By simulating Scratch's main input methods, tests can control the program under test.
        The goal is to simulate a user interacting with the program.
        The possible input includes mouse movement, mouse button presses, keyboard key presses, and entering answers to \texttt{ask} blocks.
        Inputs, that press (or release) a key or button, can specify a duration, after which the key is released (or pressed) again.
        Inputs can be performed immediately, or registered to be performed after a delay during the next execution.
        Inputs, that aren't performed during a run, will still trigger hats, that respond to the input.
        Tests can also query the current mouse position and if a certain key or button is pressed.

        Note that inputs, that press a key for a duration, do not issue multiple key press events.
        Usually, when holding down a key, the operating system repeats that key press,
        which will trigger "when key is pressed" hats repeatedly.
        This is not the case for simulated input.
        However, keys can be simulated to be released and instantly pressed again,
        which will trigger "when key is pressed" hats.
        \begin{javascriptcode}
            /* Perform a keyboard input immediately. */
            t.inputImmediate({
                device: 'keyboard',
                key: 'right arrow',
                isDown: true,
                duration: 100 // time in ms
            });

            /* Perform a mouse input one second into the next run. */
            t.addInput(1000, { device: 'mouse', x: 100, y: 200, isDown: true });

            /* Answer an ask block two seconds into the next run. */
            t.addInput(2000, { device: 'text', text: 'some answer' });

            /* Query the current state of inputs. */
            t.getMousePos(); // {x, y}
            t.isMouseDown();
            t.isKeyDown('space');
        \end{javascriptcode}
    \item \textbf{Access Scratch's output.}
        Whisker can be used to access sprites and variables of the program.
        They are accessed through sprite and variable objects, which always have the latest attributes of the sprite or variable they reference.
        Sprite objects can be obtained by their name or by their attributes.
        The provided sprite attributes include the sprite's position, rotation, size, current costume, speech bubble text, etc..
        Sprites also include the attribute values from the last execution step,
        which can be accessed through the \texttt{old} attribute of the sprite objects.
        Additionally, sprites offer some methods for detecting collisions, and other helper methods.
        \begin{javascriptcode}
            /* Various getter methods for sprites. */
                /* Get a sprite by its name. */
            const sprite = t.getSprite('Sprite1');
                /* Get sprites that match a condition. */
            const sprites = t.getSprites(sprite => sprite.x > 100);
            const clones = sprite.getClones();
            const stage = t.getStage();

            /* Various getter methods for variables. */
            const variable = stage.getVariable('my variable');
            const variables = stage.getVariables();
            const list = sprite.getList('my list');
            const lists = sprite.getLists();

            /* Accessing sprite attributes and variable values.
             * Sprites and variables always have the latest values.
             * sprite.old has the values from the last step. */
            sprite.x;
            sprite.old.x;
            variable.value;

            /* Some of the helper methods. */
            sprite.isOriginal();
            sprite.isTouchingEdge();
            sprite.isTouchingSprite(otherSprite);
        \end{javascriptcode}
    \item \textbf{Register Callbacks.}
        Tests can execute code during the program execution by registering callbacks, which get called every time the program renders a new frame.
        The can be registered to be executed before or after each frame.
        This allows tests to react to information, which the user would normally see, while the program is running.
        Callbacks can, for example, be used to track information, to perform inputs according to sprite information, or to cancel a run.
        Callbacks can be enabled and disabled.
        \begin{javascriptcode}
            /* Add a callback to be called before each step. */
            const callback = t.addCallback(() => {
                if (sprite.x > 100) {
                    t.inputImmediate({ device: 'mouse', isDown: true });
                } else if (sprite.x < 0) {
                    t.cancelRun();
                }
            });

            /* Add a callback to be called after each step (true as 2nd parameter). */
            t.addCallback(() => someList.push(sprite.x), true);

            /* Enable / disable a callback, check if a callback is active. */
            callback.disable();
            callback.enable();
            callback.isActive();
        \end{javascriptcode}
    \item \textbf{React to sprite movements and visual changes.}
        In addition to callbacks, which get called before or after each Scratch program step,
        tests can also set a function to be called whenever a sprite moves, or whenever a sprite's visuals change.
        This can be useful for detecting events, that happen during the step, but are not displayed to the user,
        which is needed for some edge cases.
        For example, let's say we test a program with two scripts:
        One of the scripts moves a sprite around.
        The other script moves the sprite back to the center of the screen whenever it touches on of the stage's edges.
        Therefore, whenever the sprite touches an edge, it is immediately moved to the center,
        even before the image is rendered.
        A normal callback will never detect the sprite touching an edge,
        but this way, detecting these collisions is possible.
        \begin{javascriptcode}
            let touchedEdge = false;

            t.onSpriteMoved(sprite => {
                touchedEdge = touchedEdge || sprite.isTouchingEdge();
            });

            t.onSpriteVisualChange(sprite => { ... });
        \end{javascriptcode}
    \item \textbf{Register constraints.}
        By registering constraints, tests can define conditions that must always hold.
        Constraints are realized through special callbacks, which perform one or more assertions.
        Whisker can be configured to fail the test on a constraint failure (the default) or to simply disable the failed constraint.
        In the second case, it is up to the test code to check which constraints failed and react accordingly.
        \begin{javascriptcode}
            /* Configure the action taken when a constraint fails. */
            t.onConstraintFailure('fail');
            t.onConstraintFailure('nothing');

            /* Check if some sprite is always visible
             * and on the right side of the screen. */
            const constraint = t.addConstraint(() => {
                t.assert.ok(sprite.visible === true, 'Sprite must always be visible.');
                t.assert.ok(sprite.x > 0, 'Sprite must always be on the right side.');
            });

            /* Enable / disable a constraint, check if a constraint is active. */
            constraint.disable();
            constraint.enable();
            constraint.isActive();
        \end{javascriptcode}
    \item \textbf{Perform randomly generated inputs.}
        Whisker provides random input generation by performing a randomly chosen input from a pool of possible inputs
        in a specified interval during program executions.
        Tests can either provide the pool or inputs to choose from themselves
        or they can let Whisker determine what inputs the program reacts to through static analysis.
        If a input consists of multiple events (i.e. if it has a duration) the same input cannot be chosen again while still active.
        If all of the inputs in the pool are active, no random inputs are performed until at least one input is done.

        Inputs in the pool can specify intervals for their attributes,
        from which a value is randomly chosen when the input is performed.
        They can also be assigned a weight to make them more likely or less likely to be chosen.
        \begin{javascriptcode}
            /* Set the interval, how often random inputs should be performed. */
            t.setRandomInputInterval(150);

            /* Register random inputs for the pool manually. */
            t.registerRandomInputs([
                  { device: 'keyboard', key: 'left arrow', duration: [50, 100] },
                  { device: 'keyboard', key: 'right arrow', duration: [50, 100] },
                  { device: 'mouse', x: [-100, 100], y: [-100, 100], weight: 0.5 }
            ]);

            /* Let Whisker detect inputs for the pool. */
            t.detectRandomInputs({ duration: [50, 100] });
        \end{javascriptcode}
\end{itemize}
\parspace


% \noindent Despite being possible in theory, Whisker does not provide the means to execute single scripts or blocks of the program directly.
% \mnote{TODO: really write about grading here?}
% It only allows executing the program as a whole.
% This has two reasons.
% Since the main goal for this testing approach is automated grading, one test suite is possibly executed on a large number of different implementations.
% Therefore, we don't want to concern ourselves with the internals of the program, since they may change from program to program.
% Secondly, executing single scripts could also lead to unexpected behaviour in the program, because scripts in typical Scratch programs often depend on other scripts, which run in parallel.
% This is mainly due to the game-like nature of usual Scratch programs.
% They often feature multiple sprites running loops in order to be interactive.

\section{Challenges of this Approach}
\label{sec:appraoch_challenges}

% \begin{itemize}
%     \item This approach allows to test programs with most of Scratch's functionality.
%         Apart from sounds and extensions, anything, which Scratch has to offer, can be tested with this approach.
%         In contrast, ITCH's previous testing approach limited Scratch programs to textual IO.
%     \item Tests are easily understood, because they control the program like a normal person would.
%     \mnote{Can't state this without proof $\rightarrow$ "we believe" or similar}
%         This is important, because students, whose programs are supposed to be tested later,
%         could be allowed to run tests on their programs themselves during development.
%         This way, students could easily receive valuable feedback about the correctness of their implementations.
%         Therefore, it is beneficial to use tests, whose actions and purposes are easily understood by students.
% \end{itemize}

This section highlights various challenges that one can face when using this approach.%
\parspace

\textbf{General testing challenges.}
Firstly, automated dynamic testing in general introduces some challenges.
For one thing, programs have to be well specified, since testing relies on the program's specification.
Therefore, if the specification is too vague, testing can become difficult.
Tests for imprecise specifications potentially need to consider more possible cases of how the program could behave.
Likewise, conflicting interpretations of the specification between the test and the program may result in false negative test outcomes.
Also, since we only test the program as a whole, testing a single part of the program can be difficult.
If some feature of the program under test depends on another feature being implemented correctly,
but the latter does not work, the first feature can not be tested properly.
\parspace

\textbf{Addressing sprites and variables.}
In order to access information about sprites and variables, they need to be addressed in some way.
Sprites can be addressed by their name or by any of their attributes, for example by their position.
Nevertheless, if a sprite, that is needed for the test, can not be found, the test can not work properly.
This can be a problem if a Scratch program deviates from the specification.
For example, if students are given a template with sprite names already in place,
some students could rename the sprites form the template, causing tests to fail.
However, errors like this can easily be detected through test reports.
% Probably the easiest way to handle this is to give students a Scratch project with the sprites and variables already in place
% as a template so the sprites and variables have the same name in each student solution.
\parspace

\textbf{Missing initialization.}
Another challenge are programs with missing initialization which are saved in a bad state.
When the green flag is pressed, the Scratch GUI simply picks up where the last execution left off.
Therefore, students might not initialize their program properly,
which can cause it to sometimes behave incorrectly when the green flag is pressed.
A program with missing initialization may be saved in a bad state,
meaning the program will behave incorrectly the next time it is started.
Since we restore the same state in the beginning of each test case,
such a program will always behave incorrectly during the tests although the implementation might be mostly correct.
\parspace

\textbf{Time-displaced events.}
Programs may have events that are supposed to trigger certain actions.
For example, a sprite touching another sprite may cause a variable to be incremented.
But the exact timing for the triggered action is often unknown, because the event is detected through a loop.
Therefore, tests have to check if the action happens in a time interval after the event occurred.
This may be done by tracking timestamps of the event and the action, and comparing these timestamps,
but this can be expensive to do.
\parspace

\chapter{Using Constraints For Flexible Test Inputs}
\label{cha:using_constraints_for_flexible_test_inputs}

This section will describe why separating control of the program under test from the test code can be beneficial,
and how this can be achieved by defining constraints, that are checked in the background.

\section{Input-Independent / Constraint-only Tests}
\label{sec:input_independent_constraint_only_tests}

Usually, tests will provide the program under test with inputs and check the resulting outputs.
However, in many cases, a different approach is possible as well.
Tests may use other sources of input and simply observe if the program's output is correct for the input provided by the source.
QuickCheck~\cite{quickcheck} by Claessen et al., for example, uses this principle, to test the correctness of Haskell programs.
In order to do so, tests define conditions, which the program must comply with.
QuickCheck then automatically generates input for the program and checks if the defined conditions hold.
\parspace

Scratch programs can often be tested in a similar way.
But in order to be able to do this, tests have to be made independent of the simulated input on the program.
This will not just enable us to test with generated input, but with other input sources as well.
For example, the program could be manually controlled by a person, or input could be recorded and played back.
Whisker also offers its own implementation of automated input generation.
% Whisker also offers its own method of automated input generation by randomizing a set of inputs,
% which are either provided by the test itself, or deduced by analyzing the program code (see section~\ref{sec:automated_input_generation} for more information on this).
\parspace

% Scratch tests can often be made independent of inputs, and Whisker provides the means to do this.
Using constraints allows us to define conditions, which the program must hold.
Constraints check the program's compliance to the conditions by continually performing assertions throughout the program execution.
Since this is done entirely in the background while the program under test is running, we can define constraints and just let the program run for some time.
This way, it does not matter in what state the program is during its execution, or what inputs it is receiving.
If a condition does not hold, the respective constraint fails.

\section{Testing Procedure}
\label{sec:constraint_testing_procedure}

Figure~\ref{fig:input_independent_testing_procedure} shows a testing procedure, which uses the aforementioned approach to test independently of the simulated input.
We will go through each step of the procedure with the aid of an example.
Consider a program with a single sprite, which is supposed to move to the right when, and only when, the right arrow key is pressed.
We want to write a test to check if the sprite's movement works correctly.
For the sake of simplicity we are going to ignore the case of the sprite not being able to move right when touching the right edge of the screen.

\begin{figure}[htpb]
    \centering
    \tikzset{>=latex,
           arrow/.style={draw, -{Latex[length=1.5mm, width=1.5mm]}},
             box/.style={draw, text width=4.3cm, minimum height=0.7cm, text centered, rounded corners},
             num/.style={draw, circle, inner sep=0.6mm, text centered},
               h/.style={fill=blue!10}}

    \begin{tikzpicture}[scale=0.9, every node/.style={scale=0.9}]
        \node[box]    at (0.2, 5.75) (initialize)  {(Give the program some time to Initialize)};
        \node[box, h] at (0.2, 4.25) (tracking)    {Setup information tracking};
        \node[box, h] at (0.2, 3.0)  (constraints) {Register constraints};
        \node[box]    at (0.2, 2.0)  (inputs)      {(Simulate Inputs)};
        \node[box]    at (0.2, 1.0)  (run)         {Run the program};
        \node[box, h] at (0.2, 0.0)  (verify)      {Verify tested situation};

        \node[num] at (-2.6, 5.75) (one)   {1};
        \node[num] at (-2.6, 4.25) (two)   {2};
        \node[num] at (-2.6, 3.0)  (three) {3};
        \node[num] at (-2.6, 2.0)  (four)  {4};
        \node[num] at (-2.6, 1.0)  (five)  {5};
        \node[num] at (-2.6, 0.0)  (six)   {6};

        \draw[arrow]
               (0.2,  6.6)
            -- (initialize)
            -- (tracking)
            -- (constraints)
            -- (inputs)
            -- (run)
            -- (verify)
            -- (0.2, -0.8);
    \end{tikzpicture}

    \caption{Input-independent / constraint-only testing procedure}
    \label{fig:input_independent_testing_procedure}
\end{figure}

\begin{enumerate}
    \item[(1)] \textbf{Give the program some time to initialize.}
        Before setting up any tracking or registering constraints we may need to run the program for a short time.
        Otherwise the initialization of the program might be tracked, which will lead to wrong test results.
        \begin{javascriptcode}
            t.runForTime(100);
        \end{javascriptcode}
    \item[(2)] \textbf{Setup information tracking.}
        Since the sprite might not move right in the same step the right arrow key is pressed,
        we want to accept slightly delayed movement in the test.
        For this purpose, we track the timestamps when the key starts being pressed,
        when the key was last pressed, and when the sprite last moved.
        We also record if the right arrow key was pressed (and released) at all,
        so we can verify it being pressed (and released) in the end of the test.
        \begin{javascriptcode}
            /* When the last right arrow key press started. */
            let startPressedTime = undefined;
            /* When the right arrow key was last recorded being pressed. */
            let pressedTime = undefined;
            /* When the sprite was last recorded moving right. */
            let movedRightTime = undefined;
            /* If the right arrow key was pressed in the previous step. */
            let previouslyPressed = t.isKeyDown('right arrow');
            /* If the right arrow key was pressed (and released) at all. */
            let pressed = false, released = false;

            const trackRightKeyCb = t.addCallback(() => {
                const currentTime = t.getTotalTimeElapsed();
                if (t.isKeyDown('right arrow')) {
                    pressedTime = currentTime;
                    if (!previouslyPressed) {
                        startPressedTime = currentTime;
                    }
                    previouslyPressed = true;
                    pressed = true;
                } else {
                    previouslyPressed = false;
                    released = true;
                }
            });

            const trackSpriteMoveCb = t.addCallback(() => {
                if (sprite.x > sprite.old.x) {
                    movedRightTime = t.getTotalTimeElapsed();
                }
            });
        \end{javascriptcode}
    \item[(3)] \textbf{Register constraints.}
        Since we tracked when the time when the key was pressed and when the sprite last moved,
        we add a constraint that simply compares the timestamps to determine if the sprite moved at the correct times.
        We chose to allow a delay of $100~\text{ms}$ for the sprite to react to the right arrow key being pressed or being released.
        We also register a second constraint, that checks that the sprite only ever stays still or moves to the right,
        but never moves in any other direction.
        \begin{javascriptcode}
            const moveTimingsConstraint = t.addConstraint(() => {
                const currentTime = t.getTotalTimeElapsed();
                if (currentTime > startPressedTime + 100) {
                    t.assert.ok(Math.abs(movedRightTime - pressedTime) <= 100,
                        'Sprite must move right when, and only when, the right arrow key is pressed.');
                }
            });

            const moveDirectionConstraint = t.addConstraint(() => {
                t.assert.ok(sprite.x >= sprite.old.x,
                    'Sprite must only stand still or move right');
                t.assert.ok(sprite.y == sprite.old.y,
                    'Sprite must not move vertically.');
            });
        \end{javascriptcode}
    \item[(4,5)] \textbf{Simulate Inputs, Run the program.}
        Now we may register inputs if we want to, and then we can run the program.
        How we simulate inputs, or if we perform inputs manually, does not matter for the test, of course.
        \begin{javascriptcode}
            t.setRandomInputInterval(250);
            t.detectRandomInputs();

            await t.runForTime(1000);
        \end{javascriptcode}
    \item[(6)] \textbf{Verify tested situation.}
        If our source of input does not guarantee that the right arrow key gets pressed at all, we risk the chance of a false positive test result,
        since the test never expects the sprite to move right.
        A similar problem occurs when the key is always pressed.
        Therefore, we want to skip the test if either the right arrow key was never pressed,
        or was pressed the whole time.
        \begin{javascriptcode}
            t.assume.ok(pressed, 'Right arrow key must be pressed.');
            t.assume.ok(released, 'Right arrow key must be released.');
        \end{javascriptcode}
\end{enumerate}
\parspace

Figure~\ref{fig:normal_input_independent_test_comparison} shows the resulting test code for the example
and compares it to a similar test, which simulates inputs deliberately to check the sprite's movement.
We can easily see, that this approach takes quite a bit more code than the more normal approach.
But at the same time, the input-independent testing procedure is able to scale better than the other testing procedure.
Once enough tracking is set up, writing constraints becomes easy and requires little code.
At the same time, because the constraints are isolated from the program execution itself,
many constraints can possibly be combined into a single test.
For this purpose, Whisker offers an option to disable failed constraints instead of failing the entire test.
In this case, tests can simply check which constraints failed, and let Whisker generate a test report from them.

\begin{listing}[htpb]
    \centering
    \begin{subfigure}[b]{.40\textwidth}
        \centering
        \begin{minted}[autogobble, breaklines, linenos, fontsize=\tiny, framesep=2mm, frame=lines]{javascript}
            const sprite = t.getSprite('Sprite1');

            /* Give the program some time to initialize. */
            await t.runForTime(100);

            let oldX = sprite.x;

            /* Check that the sprite never moves into an other
             * direction that right. */
            t.addConstraint(() => {
                t.assert.ok(sprite.x >= sprite.old.x,
                    'Sprite must only stand still or move right');
            });

            /* Run without inputs and check that the sprite
             * doesn't move. */

            await t.runForTime(1000);

            t.assert.ok(oldX === sprite.x,
                'Sprite must not move right when no key is pressed.');

            /* Run with the right arrow key being pressed and
             * check that the sprite moves to the right. */

            t.inputImmediate({
                device: 'keyboard',
                key: 'right arrow',
                isDown: true
            });

            await t.runForTime(1000);

            t.assert.ok(oldX < sprite.x,
                'Sprite must move right when right arrow key is pressed.');
        \end{minted}
        \vspace{-\bigskipamount}
        \caption{Normal test}
    \end{subfigure}
    \hspace{.08\textwidth}
    \begin{subfigure}[b]{.50\textwidth}
        \centering
        \begin{minted}[autogobble, breaklines, linenos, fontsize=\tiny, framesep=2mm, frame=lines]{javascript}
            let sprite = t.getSprite('Sprite1');

            /* (1) Give the program some time to initialize. */
            await t.runForTime(100);

            /* When the last right arrow key press started. */
            let startPressedTime = undefined;
            /* When the right arrow key was last recorded being pressed. */
            let pressedTime = undefined;
            /* When the sprite was last recorded moving right. */
            let movedRightTime = undefined;
            /* If the right arrow key was pressed in the previous step. */
            let previouslyPressed = t.isKeyDown('right arrow');
            /* If the right arrow key was pressed (and released) at all. */
            let pressed = false, released = false;

            /* (2) Track when the right arrow key is being pressed,
             * and when the sprite is moving to the right. */

            const trackRightKeyCb = t.addCallback(() => {
                const currentTime = t.getTotalTimeElapsed();
                if (t.isKeyDown('right arrow')) {
                    pressedTime = currentTime;
                    if (!previouslyPressed) {
                        startPressedTime = currentTime;
                    }
                    previouslyPressed = true;
                    pressed = true;
                } else {
                    previouslyPressed = false;
                    released = true;
                }
            });

            const trackSpriteMoveCb = t.addCallback(() => {
                if (sprite.x > sprite.old.x) {
                    movedRightTime = t.getTotalTimeElapsed();
                }
            });

            /* (3) Check if the sprite only moves when the right arrow
             * key was pressed, and if it doesn't move when the key was
             * not pressed. */

            const moveTimingsConstraint = t.addConstraint(() => {
                const currentTime = t.getTotalTimeElapsed();
                if (currentTime > startPressedTime + 100) {
                    t.assert.ok(Math.abs(movedRightTime - pressedTime) <= 100,
                        'Sprite must move right when, and only when, the right arrow key is pressed.');
                }
            }));

            const moveDirectionConstraint = t.addConstraint(() => {
                t.assert.ok(sprite.x >= sprite.old.x,
                    'Sprite must only stand still or move right');
                t.assert.ok(sprite.y == sprite.old.y,
                    'Sprite must not move vertically.');
            }));

            /* (4) Some code, which registers inputs. Or nothing if
             * inputs are done manually. For example, generated input: */
            t.setRandomInputInterval(250);
            t.detectRandomInputs();

            /* (5) Run the program. */
            await t.runForTime(5000);

            /* (6) Check if the right arrow key was pressed at all,
             * and if it was ever released. */
            t.assume.ok(pressed, 'Right arrow key must be pressed.');
            t.assume.ok(released, 'Right arrow key must be released.');
        \end{minted}
        \vspace{-\bigskipamount}
        \caption{Input-independent test}
        \label{fig:normal_input_independent_test_comparison_constraint}
    \end{subfigure}
    \caption{Comparison of normal tests and an input-independent tests}
    \label{fig:normal_input_independent_test_comparison}
\end{listing}

\section{Resetting the Program During a Test}
\label{sec:resetting_the_program_during_a_test}

When testing with randomly generated input, programs often need to be reset multiple times inside of one test execution,
because the test could get stuck in some state of the program.
\parspace

To reset a Scratch program, we may simply press the green flag button again,
but we also need to consider the tracked information as well as the registered callbacks and constraints when resetting the program under test.
Any per-run information we tracked has to be cleared, and callbacks as well as constraints have to disabled before resetting the program.
After resetting the program, callbacks and constraints have to be enabled again.
Also, if the test configured Whisker not to fail the test when a constraint fails, then the test needs to check which constraints failed before disabling them
and then only enable constraints which were active before.
Figure~\ref{fig:resetting_the_program_under_test_procedure} shows a procedure that may be used to reset the program under test.
This procedure replaces step $5$ (Run the program) in the input-independent testing procedure (see Figure~\ref{fig:input_independent_testing_procedure}).
Additionally the code example in Listing~\ref{fig:resetting_the_program_under_test_example}
implements this procedure for the example from the previous section.
\parspace

Sometimes, information may need to be tracked across resets in order to determine what events occurred in the program during the whole test execution.
This information may be needed to rule out constraints, whose situation never occurred, after the test execution.
An example for this are the variables \texttt{pressed} and \texttt{released} in Listing~\ref{fig:normal_input_independent_test_comparison_constraint}.
\parspace


\begin{figure}[htpb]
    \centering
    \begin{subfigure}[m]{.3\textwidth}
        \centering
        \tikzset{>=latex,
               arrow/.style={draw, -{Latex[length=1.5mm, width=1.5mm]}},
                 box/.style={draw, text width=4.3cm, minimum height=0.7cm, text centered, rounded corners},
                 num/.style={draw, circle, inner sep=0.6mm, text centered},
                   h/.style={fill=blue!10}}

        \begin{tikzpicture}[scale=0.9, every node/.style={scale=0.9}]
            \node[box, h] at (0.2,  6.75) (suspendcb)  {Suspend Information Tracking};
            \node[box, h] at (0.2,  5.50) (suspendcn)  {Suspend Constraints};
            \node[box, h] at (0.2,  4.25) (resetinfo)  {Reset per-run information};
            \node[box]    at (0.2,  2.25) (greenflag)  {Press the green flag (and give the program some time to Initialize)};
            \node[box]    at (0.2,  0.75) (spriteref)  {Get new sprites};
            \node[box, h] at (0.2, -0.75) (activatecb) {Activate Information Tracking};
            \node[box, h] at (0.2, -2.00) (activatecn) {Activate Constraints};
            \node[box]    at (0.2, -3.00) (run)        {Run the program};


            \node[num] at (-2.6,  6.75) (one)   {1};
            \node[num] at (-2.6,  5.50) (two)   {2};
            \node[num] at (-2.6,  4.25) (three) {3};
            \node[num] at (-2.6,  2.25) (four)  {4};
            \node[num] at (-2.6,  0.75) (five)  {5};
            \node[num] at (-2.6, -0.75) (six)   {6};
            \node[num] at (-2.6, -2.00) (seven) {7};
            \node[num] at (-2.6, -3.00) (eight) {8};

            \draw[arrow]
                   ( 0.2,  7.6)
                -- (suspendcb)
                -- (suspendcn)
                -- (resetinfo)
                -- (greenflag)
                -- (spriteref)
                -- (activatecb)
                -- (activatecn)
                -- (run)
                -- ( 0.2, -3.8);

            % \draw[-, rounded corners]
            %        ( 0.2, -2.8)
            %     -- (-3.4, -2.8)
            %     -- (-3.4,  7.7)
            %     -- ( 0.2,  7.7);
        \end{tikzpicture}
        \caption{Procedure for resetting the program under test}
        \label{fig:resetting_the_program_under_test_procedure}
    \end{subfigure}%
    \hspace{.13\textwidth}%
    \begin{subfigure}[m]{.55\textwidth}
        \centering
        \begin{minted}[autogobble, breaklines, linenos, fontsize=\scriptsize, framesep=2mm, frame=lines]{javascript}
            for (let i = 0; i < 5; i++) {
                const constraints = [moveTimingsConstraint,
                    moveDirectionConstraint];

                /* (1) Suspend information tracking. */
                trackRightKeyCb.disable();
                trackSpriteMoveCb.disable();

                /* (2) Suspend constraints.
                 * Get a list active constraints first.
                 * (Constraints that did not fail) */
                const ac = constraints.filter(c => c.isActive());
                for (const constraint of ac) {
                    constraint.disable();
                }

                /* (3) Reset per-run information
                   (keep values of 'pressed' and 'released'). */
                startPressedTime = undefined;
                pressedTime = undefined;
                movedRightTime = undefined;
                previouslyPressed = t.isKeyDown('right arrow');

                /* (4) Press the green flag and give the program
                   some time to Initialize. */
                t.greenFlag();
                await t.runForTime(100);

                /* (5) Get new sprites. */
                let sprite = t.getSprite('Sprite1');

                /* (6) Activate information tracking. */
                trackRightKeyCb.enable();
                trackSpriteMoveCb.enable();

                /* (7) Activate constraints. */
                for (const constraint of ac) {
                    constraint.enable();
                }

                /* (8) Run the program. */
                await t.runForTime(2000);
            }
        \end{minted}
        \vspace{-\bigskipamount}
        \caption{Example code for resetting the program under test (extends Listing~\ref{fig:normal_input_independent_test_comparison_constraint})}
        \label{fig:resetting_the_program_under_test_example}
    \end{subfigure}

    \caption{Resetting the program under test}
    \label{fig:resetting_the_program_under_test}
\end{figure}


% Register Callbacks for Information Tracking (per run for constraints and across runs to rule out constraints in the end)
% Register Constraints
%
% repeat:
%     Suspend Information Tracking Callbacks
%     Suspend Constraints
%
%     Reset Per Run Information
%
%     Press Green Flag
%     Run the Program For Short Time to Initialize It
%
%     Activate Information Tracking Callbacks
%     Activate Constraints
%
%     Run the Program
%
% Check Across-Run Information to Disable Constraints, whose situation to check did not occur
% Generate test report


% \begin{figure}[htpb]
%     \centering
%     \begin{subfigure}[b]{.45\textwidth}
%         \centering
%         \tikzset{>=latex,
%                  box/.style={draw, text width=4.3cm, minimum height=0.7cm, text centered, rounded corners},
%                  num/.style={draw, circle, inner sep=0.6mm, text centered},
%                    h/.style={fill=blue!10}}
%
%         \begin{tikzpicture}
%             \node[box] at ( 0.2,  3.0) (run)        {Run the program};
%             \node[box] at ( 0.2,  2.0) (inputs)     {Simulate inputs};
%             \node[box] at ( 0.2,  1.0) (checks)     {Perform checks};
%
%             \node[num] at (-2.6,  3.0) (one)   {1};
%             \node[num] at (-2.6,  2.0) (two)   {2};
%             \node[num] at (-2.6,  1.0) (three) {3};
%
%             \draw[->]
%                    ( 0.2,  4.8)
%                 -- (run)
%                 -- (inputs)
%                 -- (checks)
%                 -- ( 0.2, -1.0);
%
%             \draw[shorten >= 2pt, rounded corners, dashed, ->]
%                    ( 0.2,  0.0)
%                 -- (-3.4,  0.0)
%                 -- (-3.4,  4.0)
%                 -- ( 0.2,  4.0);
%         \end{tikzpicture}
%
%         \caption{Normal Test Procedure}
%         \label{fig:normal_test_procedure}
%     \end{subfigure}
%     \begin{subfigure}[b]{.45\textwidth}
%         \centering
%         \tikzset{>=latex,
%                  box/.style={draw, text width=4.3cm, minimum height=0.7cm, text centered, rounded corners},
%                  num/.style={draw, circle, inner sep=0.6mm, text centered},
%                    h/.style={fill=blue!10}}
%
%         \begin{tikzpicture}
%             \node[box] at ( 0.2,  4.25) (tracking)    {Setup tracking of information};
%             \node[box] at ( 0.2,  3.0)  (constraints) {Register constraints};
%             \node[box] at ( 0.2,  2.0)  (run)         {Run the program};
%             \node[box] at ( 0.2,  1.0)  (inputs)      {(Simulate Inputs)};
%             \node[box] at ( 0.2,  0.0)  (filter)      {Filter constraints};
%
%             \node[num] at (-2.6,  4.25) (one)   {1};
%             \node[num] at (-2.6,  3.0)  (two)   {2};
%             \node[num] at (-2.6,  2.0)  (three) {3};
%             \node[num] at (-2.6,  1.0)  (four)  {4};
%             \node[num] at (-2.6,  0.0)  (five)  {5};
%
%             \draw[->]
%                    (tracking)
%                 -- (constraints)
%                 -- (run)
%                 -- (inputs)
%                 -- (filter)
%                 -- ( 0.2, -1.0);
%         \end{tikzpicture}
%
%         \caption{Constraint-only Test Procedure}
%         \label{fig:constraint_only_test_procedure}
%     \end{subfigure}
%     \caption{Comparison of the procedure of normal tests and constraint-only tests}
%     \label{fig:comparison_of_the_procedure_of_normal_tests_and_constraint_only_tests}
% \end{figure}


% REMEMBER: Write the thesis from the view of the reader. How would I like to READ the thesis?

% FROM APPROACH:
% - Before starting a test case, the project is loaded to make sure that every test starts with the same program state
%     - If this would not be done, it would be very hard to test programs that don't do proper initialization
%     - Tests would depend on the state the previous test leaves the program in
%     $\rightarrow$ Very inconsistent, therefore desirable to load the program before each test

%         - This work will explore automatic generation of input for Scratch programs
%         - Input can be generated randomly
%         - Static analysis can be used to detect the input events the program recognizes and to give the program inputs
%         - A combination of both will be used for this work

\chapter{Implementation}

TODO: keyword "automation"

- This section shows a implementation of a testing utility for Scratch
- Uses the approach described above
- Implemented in JavaScript (ES6) for compatibility with Scratch 3.0, which is implemented in JavaScript

\section {Scratch 3.0 Implementation Details}

Scratch's interpreter sequentializes the execution.
This is necessary for the single-threaded JavaScript environment Scratch is run in.
This means that no race-conditions can occur on a language level.

\section{General Design}

- TODO explain Scratch's design?
- TODO explain why it's possible to execute code before and after Scratch's step without interfering with the Scratch program

- Tests are written in JavaScript
- Usually one test is a function of a .js file

- Does not change the virtual machine in any way
    $\rightarrow$ designed so it can be used with any instance of the Scratch virtual machine
    $\rightarrow$ same version of Whisker can be used with different versions of the Scratch virtual machine
    $\rightarrow$ even something like a browser addon for the original Scratch page would be possible

- Whisker is designed to be a layer between test code and the scratch virtual machine
- Allows to interact with the VM in a test-friendly way
- The main class VM Wrapper and its components make up a wrapper around the scratch virtual machine,
  which offers extra functionality for testing
- Test Driver offers a user-friendly interface between the test code and the VM Wrapper
- Test uses Test Driver to simulate input, get information about sprites

- Test Driver could be acquired through a helper method or passed to the test method
- Up to the testing framework to provide the test driver
- The examples in this chapter will refer to the test driver with the variable \texttt{t}

\begin{listing}[ht]
    \centering
    \begin{minipage}[t]{.45\textwidth}
        \begin{javascriptcode}
            async function test (t) {
                ...
            }
        \end{javascriptcode}
        \vspace{-\bigskipamount}
        Getting the test driver as a parameter to the test function
    \end{minipage}
    ~
    \begin{minipage}[t]{.45\textwidth}
        \begin{javascriptcode}
            async function test () {
                const t = acquireTestDriver();
            }
        \end{javascriptcode}
        \vspace{-\bigskipamount}
        Getting the test driver through some helper method
    \end{minipage}
    \caption{Examples of how to acquire the test driver}
    \label{fig:examples_of_how_to_acquire_the_test_driver}
\end{listing}

=== Limitations
- Scratch depends on the renderer
    - Some functionality of the Scratch virtual machine depends on the renderer
    - Headless tests are impossible without restricting the Scratch program to a subset of available blocks
- Therefore Whisker assumes the VM is always run with a renderer in place

\begin{figure}[ht]
    \centering
    \tikzset{>=latex,
             label/.style={draw=none, text width=5.3cm, minimum height=0.5cm, text centered},
               box/.style={draw,      text width=2.5cm, minimum height=0.7cm, text centered, rounded corners},
                 h/.style={fill=blue!10}}

    \begin{tikzpicture}
        \node[box]   at ( 0.0,  3.0) (testcode)      {Test Code};
        \node[box]   at ( 0.0,  1.5) (testdriver)    {Test Driver};
        \node[label] at ( 0.0,  0.0) (vmwrapper)     {VM Wrapper};
        \node[box]   at (-1.4, -0.7) (sprites)       {Sprites};
        \node[box]   at (-1.4, -1.6) (inputs)        {Inputs};
        \node[box]   at ( 1.5, -0.7) (callbacks)     {Callbacks};
        \node[box]   at ( 1.5, -1.6) (constraints)   {Constraints};
        \node[box]   at (-2.0, -3.2) (scratchvm)     {Scratch VM};
        \node[box]   at ( 2.2, -3.2) (scratchrender) {Renderer};

        \begin{scope}[on background layer]
            \node[draw, h, rounded corners, fit=(vmwrapper)(sprites)(inputs)(callbacks)(constraints)] (container) {};
        \end{scope}

        \foreach \pp/\pf/\pt in {--/testcode/testdriver,
                                 --/testdriver/container,
                                 --/container/scratchvm,
                                 --/container/scratchrender,
                                 --/scratchvm/scratchrender}
        \draw[shorten >= 2pt, ->] (\pf) \pp (\pt);

        % \draw[shorten >= 2pt, rounded corners, dashed, ->]
        %        (constraints)
        %     -- ( 3.5, -1.6)
        %     -- ( 3.5,  3.0)
        %     -- (testcode);
        % \draw[dashed, -] (callbacks) -- ( 3.5, -0.7);
    \end{tikzpicture}

    \caption{Components of Whisker}
    \label{fig:components_of_whisker}
\end{figure}

\section{The Step Loop}

- The core of the Scratch virtual machine is a step-function, which is called at a constant interval (using JavaScript's \texttt{setInterval()})
- Interval of 30 times / second for Scratch 2.0, 60 times / second for Scratch 3.0
- The function executes the program until a time limit is reached and then redraws the scene
- If some visual change occurs in the project, the program execution is stopped earlier and the scene is rendered

\begin{listing}[ht]
    \centering
    \begin{javascriptcode}
        STEP_TIME = 1000 / STEPS_PER_SECOND;
        WORK_TIME = 0.75 * STEP_TIME;

        while (running &&
               timeElapsed < WORK_TIME &&
               !redrawRequested) {
            for (thread of threads) {
                stepThread(thread);
            }
        }

        renderer.draw();
    \end{javascriptcode}
    \vspace{-\bigskipamount}
    \caption{Simplified Scratch Step Procedure}
    \label{fig:simplified_scratch_step_procedure}
\end{listing}

- Instead of executing the step function via interval, Whisker executes its own step loop, which calls Scratch's step function
- Before and after the step of the Scratch program, test code is run, registered inputs are performed and sprite objects are updated

- This should either not affect the Scratch program, or only affect it minimally
    - Renderer might not need to use the entire allocated rendering time
    - If something changes in a Scratch program, usually a sprite moves $\rightarrow$ Scratch will only use a fraction of the entire allocated work time in most cases
    - Scratch uses real time to track wait times $\rightarrow$ not affected much by a step that takes longer that normally

=== Scratch programs depend on real time
- Possible problem: additional computations could cause the program to run slower
- Scratch blocks that involve timings, like "wait" or "say for secs" use real time to delay execution
$\rightarrow$ If the additional computations take too long, they could influence the program

\begin{figure}[ht]
    \centering
    \tikzset{>=latex,
             box/.style={draw, text width=4.3cm, minimum height=0.7cm, text centered, rounded corners},
             num/.style={draw, circle, inner sep=0.6mm, text centered},
               h/.style={fill=blue!10}}

    \begin{tikzpicture}
        \node[box]    at ( 0.2,  5.0) (callbacksbefore) {Call Callbacks (before)};
        \node[box]    at ( 0.2,  4.0) (inputs)          {Perform Inputs};
        \node[box]    at ( 0.2,  3.0) (sprites)         {Update Sprites};
        \node[box, h] at ( 0.2,  2.0) (step)            {Step Scratch Program};
        \node[box]    at ( 0.2,  1.0) (callbacksafter)  {Call Callbacks (after)};
        \node[box]    at ( 0.2,  0.0) (constraints)     {Check Constraints};

        \node[text width=2cm] at (-4.5, 2.5) (wait)  {Wait until next step is due};

        \node[num] at (-2.6,  5.0) (one)   {1};
        \node[num] at (-2.6,  4.0) (two)   {2};
        \node[num] at (-2.6,  3.0) (three) {3};
        \node[num] at (-2.6,  2.0) (four)  {4};
        \node[num] at (-2.6,  1.0) (five)  {5};
        \node[num] at (-2.6,  0.0) (six)   {6};

        \draw[->]
               (callbacksbefore)
            -- (inputs)
            -- (sprites)
            -- (step)
            -- (callbacksafter)
            -- (constraints)
            -- ( 0.2, -1.5);

        \draw[shorten >= 2pt, rounded corners, dashed, ->]
               ( 0.2, -1.0)
            -- (-3.4, -1.0)
            -- (-3.4,  6.0)
            -- ( 0.2,  6.0)
            -- (callbacksbefore);
    \end{tikzpicture}

    \caption{Whisker Step Procedure}
    \label{fig:whisker_step_procedure}
\end{figure}

\section{The Test Interface}

=== WM Wrapper
- Control the execution of the scratch program
- Run the program until a certain amount of time has passed or a condition has been met
- Get the time elapsed since the start of the test or the start of the last run
- Cancel a run
- Uses JavaScript's Promise API to wait until a run is finished

\begin{listing}[ht]
    \centering
    \begin{javascriptcode}
        await t.runForTime(500);
        await t.runUntil(() => !t.projectRunning(), 1000);
        t.assert.ok(t.getTotalTimeElapsed() < 1000);
    \end{javascriptcode}
    \vspace{-\bigskipamount}
    \caption{Example code for the VM Wrapper}
    \label{fig:example_code_for_the_vm_wrapper}
\end{listing}

=== Sprites
- Sprite is not the same as sprite in Scratch VM
    - Explain distinguishes between sprites and targets / rendered targets
    - Sprite contains the blocks, graphics (costumes), etc.
    - Target / rendered target is an instance of the sprite
    - Whisker sees every rendered target as a sprite
    - The original target of a Scratch sprite as well as its clones are each an instance of a ''sprites''
    - TODO: explain clones here

- Sprites work by wrapping around the original
- If some getter of the sprite is called, the actual value is retrieved from the original target
- Most properties are implemented as JavaScript getters $\rightarrow$ look like properties of Whisker's sprite object

- Information about sprites and variables
- Gives the information that the test uses
- Does not allow to manipulate sprites and variables
- Contains ''old'' value for every fitting property
    - Saves the value from the last step
    - Useful for constraints (see later)
    - Initialized with the present value
- Sprites are only tracked once they are retrieved via one of the getter method
- Helps, for example, with programs that spawn a lot of clones (could pose a performance problem otherwise)


\begin{listing}[ht]
    \centering
    \begin{javascriptcode}
        const sprite = t.getSprite('Sprite1');
        const variable = sprite.getVariable('Variable1');
        const sprites = t.getSprites(s => s.x > 100);

        t.assert.equal(sprite.x, 100);
        t.assert.equal(sprite.old.x, 100);
        t.assert.equal(variable.value, 5);
    \end{javascriptcode}
    \vspace{-\bigskipamount}
    \caption{Example code for Sprites}
    \label{fig:example_code_for_sprites}
\end{listing}

=== Inputs
- INPUTS ARE SIMULATED ON THE VM, NO ACTIONS ARE SIMULATED ON THE OS LEVEL OR ANYTHING

- Simulate inputs on the program
- Can be registered to be called after a certain amount of time or be executed immediately
- Registering a Input with 0 delay is different from executing it immediately
=== Kinds of Input
- At the moment: only mouse and keyboard input
- Keyboard:
    - Press a key
    - Release a key
    - Toggle a key
    - Press / release a key for a certain duration
- Mouse (only left mouse button):
    - move cursor to position
    - move cursor to sprite (+offset)
    - press mouse button
    - release mouse button
    - toggle mouse button
    - Press / release mouse button for a certain duration

\begin{listing}[ht]
    \centering
    \begin{javascriptcode}
        t.inputImmediate({
            device: 'keyboard',
            key: 'right arrow',
            isDown: true
        });

        const mouseInput = t.addInput(1000, {
            device: 'mouse',
            x: 100,
            y: 0,
            isDown: true,
            duration: 500
        });

        t.assert.ok(t.isKeyDown('right arrow'));
        t.removeInput(mouseInput);
        t.addInput(mouseInput);
    \end{javascriptcode}
    \vspace{-\bigskipamount}
    \caption{Example code for Random Inputs}
    \label{fig:example_code_for_random_inputs}
\end{listing}

=== Random Inputs
- Provides a simple way to perform inputs randomly
- Way of testing the program without deliberately controlling the inputs
- In a set time interval (at the next step), an input is randomly selected from a pool of registered random inputs

- You  can register inputs for the random pool, or let Whisker choose inputs based on the blocks used in the program
- Random inputs can be detected through simple static analysis
$\rightarrow$ Blocks that take inputs and their options are analyzed

- TODO table of detected blocks and resulting inputs ?

\begin{listing}[ht]
    \centering
    \begin{javascriptcode}
        t.setRandomInputInterval(150);
        t.registerRandomInputs([
            {
                device: 'keyboard',
                key: 'left arrow',
                duration: [50, 100]
            },
            {
                device: 'keyboard',
                key: 'right arrow',
                duration: [50, 100]
            }
        ]);
        t.detectRandomInputs();
    \end{javascriptcode}
    \vspace{-\bigskipamount}
    \caption{Example code for Inputs}
    \label{fig:example_code_for_inputs}
\end{listing}

=== Callbacks
- Register callbacks that are called after every step
- Can be registered to run at two positions in the step cycle (see diagram later)
- Purpose:
    - Information Tracking:
        - Track events
        - e.g how many times a sprite touches some other sprite
              or how long a sprite is invisible, etc.
    - Inputs:
        - Allows performing conditional inputs
        - Control the program like a player would
        - e.g. follow a sprite with the mouse cursor

\begin{listing}[ht]
    \centering
    \begin{javascriptcode}
        const sprite = t.getSprite('Sprite1');

        const callback = t.addCallback(() => {
            t.inputImmediate({
                device: 'mouse',
                x: sprite.x,
                y: sprite.y
            });
        });

        t.runForTime(1000);

        callback.disable();
        callback.enable();
    \end{javascriptcode}
    \vspace{-\bigskipamount}
    \caption{Example code for Callbacks}
    \label{fig:example_code_for_callbacks}
\end{listing}


=== Constraints
- Describe conditions that must always hold
- Failed constraints can be configured to fail the test, stop the current program run (\texttt{run...()}), or do nothing
- Can be used to perform checks like "sprite xy is always visible when sprite yz is visible"
- Implemented as callbacks that execute assertions
    - Advantages:
        - Concise syntax
        - Multiple assertions per constraint possible
        - Assertion messages can be used to describe the constraint failure
    - Disadvantages:
        - Assertion methods have to be efficient, e.g. if assertion message is constructed every time it could be too slow
        - Need to catch exceptions if constraints should not fail the test
- Can separate assertions from the program execution
    - Define constraints for tested properties, then use whatever input (deliberate / random / manual)
    - Problem: most constraints still hold if the tested property is not implemented at all
        - e.g. constraint that checks that a sprite never moves left will hold if the sprite doesn't move at all

\begin{listing}[ht]
    \centering
    \begin{javascriptcode}
        const sprite = t.getSprite('Sprite1');

        t.onConstraintFailure('fail');

        const constraint = t.addConstraint(() => {
            t.assert.ok(sprite.x >= sprite.old.x);
        });

        t.runForTime(1000);

        constraint.disable();
    \end{javascriptcode}
    \vspace{-\bigskipamount}
    \caption{Example code for Constraints}
    \label{fig:example_code_for_constraints}
\end{listing}

\section{Coverage Measurement}
- Whisker can measure simple statement coverage
- Project wide and for single sprites / the stage

- Only measures blocks that are part of a script / connected to a hat (this includes procedure definitions)
- Other blocks are obviously not reachable

- Useful to detect if a project has been properly captured by a test
    - If coverage is low, there is a problem with the test or project
    - Or the project contains unnecessary code

- Initialized by traversing the scripts and noting each block id of the blocks in a project
- Coverage is simply measured by tracking which block ids are put on top of the stack of the threads in the Scratch virtual machine

\section{Running Tests}
\label{sec:running_tests}
- Whisker comes with an optional testing framework
- Include a sample test report?

=== Seeing test output, interactive tests
- Users will want to see the program's output while it is run
    - to check if the tests run correctly, to check if the program runs correctly
    - Difficult to determine a problem with the project from just textual test reports
    - Tests without showing output are not very useful in such an interactive environment
$\rightarrow$ Has to be able to run in a interactive environment
    - Web GUI, which can be run by any modern browser
    - Allows users to run individual tests on the project and see the program execution

=== Batch Testing of Projects
- Some tests for Scratch projects can take a long time because projects run in real time
    - raises the need to test scratch programs in parallel
    - Scratch depends on the renderer
        - Some functionality of the Scratch virtual machine depends on the renderer
        - Headless tests are impossible without restricting the Scratch program to a subset of available blocks
$\rightarrow$ Web GUI has the option to run tests on multiple projects sequentially, but this might still take a long time depending on the project and the test suite
$\rightarrow$ Electron
    - Running tests in multiple processes could circumvent the problem
    - Electron provides a renderer that can be used to render the Scratch output to
    - Spawns multiple processes which open a window each, one project is tested in each window


\begin{frame}
    \bigcenter{Evaluation}
\end{frame}

\begin{frame}
    \bigcenter{Can test results match results of manual grading?}
\end{frame}

\begin{frame}\frametitle{Evaluation, Test Results}
    Test subjects:
    \begin{itemize}
        \item \textcolor{upfim}{37 student implementations} of a simple catching game
        \item from a 6th and 7th grade Scratch workshop~\cite{keller}
        \item Graded manually, on a scale from 0 to 30
    \end{itemize}

    \pause
    \bigskip

    Two test suites:
    \begin{itemize}
        \item \textcolor{upfim}{''normal'' test suite} with 28 test cases
            \begin{itemize}
                \item each test case executes the program once, independently
            \end{itemize}
        \item \textcolor{upfim}{''constraint'' test suite} with 26 constraints
            \begin{itemize}
                \item only one test case
                \item uses generated input
                \item runs the program for 10s with 30 resets $\rightarrow$ 300s
            \end{itemize}
    \end{itemize}

    \pause
    \bigskip

    Measured item:
    \begin{itemize}
        \item Correlation between manual scores and the number of test or constraint passes
    \end{itemize}
\end{frame}

\begin{frame}\frametitle{Evaluation, Test Results}
    Excluded Projects:
    \begin{itemize}
        \item 6 of the 37 projects were excluded from the statistics
        \item Mostly excluded because they don't start correctly
        \item Would distort results, since manual grading ignored the issues
    \end{itemize}

    % TODO: note the excluded projects for potential questions
\end{frame}

\begin{frame}\frametitle{Evaluation, Test Results, Normal Tests}
    \begin{figure}
        \begin{minipage}{.85\textwidth}
            \includegraphics[width=\textwidth]{r/scatter-normal-1}
            \caption{Comparison between results of normal tests and manual scores, 1st run}
        \end{minipage}
    \end{figure}
\end{frame}

\begin{frame}\frametitle{Evaluation, Test Results, Normal Tests}
    \begin{figure}
        \begin{minipage}{.85\textwidth}
            \includegraphics[width=\textwidth]{r/scatter-normal-avg}
            \caption{Comparison between results of normal tests and manual scores, average over 10 runs}
        \end{minipage}
    \end{figure}
\end{frame}

\begin{frame}\frametitle{Evaluation, Test Results, Constraint Tests}
    \begin{figure}
        \begin{minipage}{.85\textwidth}
            \includegraphics[width=\textwidth]{r/scatter-random-1}
            \caption{Comparison between results of constraint tests and manual scores, 1st run}
        \end{minipage}
    \end{figure}
\end{frame}

\begin{frame}\frametitle{Evaluation, Test Results, Constraint Tests}
    \begin{figure}
        \begin{minipage}{.85\textwidth}
            \includegraphics[width=\textwidth]{r/scatter-random-avg}
            \caption{Comparison between results of constraint tests and manual scores, average over 10 runs}
        \end{minipage}
    \end{figure}
\end{frame}

\begin{frame}
    \bigcenter{What coverage can be achieved with automated input?}
\end{frame}

\begin{frame}\frametitle{Evaluation, Coverage of Automated Input}
    Test subjects:
    \begin{itemize}
        \item 24 sample solutions to Code Club's\footnote{\url{https://codeclubprojects.org/}} online Scratch courses
        \item Run with generated input for 10 minutes
    \end{itemize}

    \pause
    \bigskip

    Measured item:
    \begin{itemize}
        \item Mean coverage of the projects after 10 minutes
        \item Coverage measured every second
    \end{itemize}
\end{frame}

\begin{frame}\frametitle{Evaluation, Coverage of Automated Input}
    \begin{figure}
        \includegraphics[width=\textwidth]{r/coverage-bar-random-input-1}
        \caption{Coverage per project, 1st run}
    \end{figure}
\end{frame}

\begin{frame}\frametitle{Evaluation, Coverage of Automated Input}
    \begin{figure}
        \includegraphics[width=\textwidth]{r/coverage-line-random-input-1}
        \caption{Coverage over time, 1st run}
    \end{figure}
\end{frame}

\begin{frame}\frametitle{Evaluation, Coverage of Automated Input}
    \begin{figure}
        \includegraphics[width=\textwidth]{r/coverage-bar-random-input-avg}
        \caption{Coverage per project, average over 10 runs}
    \end{figure}
\end{frame}

\begin{frame}\frametitle{Evaluation, Coverage of Automated Input}
    \begin{figure}
        \includegraphics[width=\textwidth]{r/coverage-line-random-input-avg}
        \caption{Coverage over time, average over 10 runs}
    \end{figure}
\end{frame}

\begin{frame}\frametitle{Evaluation, Coverage of Automated Input}
    \begin{figure}
        \includegraphics[width=\textwidth]{r/coverage-bar-no-input-1}
        \caption{Coverage per project, 1st run}
    \end{figure}
\end{frame}

\begin{frame}\frametitle{Evaluation, Coverage of Automated Input}
    \begin{figure}
        \includegraphics[width=\textwidth]{r/coverage-line-no-input-1}
        \caption{Coverage over time, 1st run}
    \end{figure}
\end{frame}

\begin{frame}\frametitle{Evaluation, Coverage of Automated Input}
    \begin{figure}
        \includegraphics[width=\textwidth]{r/coverage-bar-no-input-avg}
        \caption{Coverage per project, average over 10 runs}
    \end{figure}
\end{frame}

\begin{frame}\frametitle{Evaluation, Coverage of Automated Input}
    \begin{figure}
        \includegraphics[width=\textwidth]{r/coverage-line-no-input-avg}
        \caption{Coverage over time, average over 10 runs}
    \end{figure}
\end{frame}

%     Show that the method is suited for answering the research questions.
%     Show that the method is valid and reliable.
%     How was the data collected, what options does this give?
%     What did I do to increase validity?
%     Describe weaknesses as well as strengths.
%     Defend and, at the same time, criticise the choices I made.
%     Document what I did and what I did not do.

%    Present findings in a systematic manner.

\chapter{Conclusion}
\label{cha:conclusion}

% In this work we described a way to perform runtime testing on Scratch programs
% by automating Scratch's IO operations.
% Firstly, we gave some background information about the topic.
% We described the Scratch language and its programming environment.
% Then we explained previous approaches that tackled the problem of testing Scratch programs.
% We examined Hairball~\cite{hairball}, which analyzes Scratch programs by performing static analysis on them,
% as well as ITCH~\cite{itch}, which transforms Scratch programs to automate \texttt{ask} and \texttt{say} blocks.
% We described various general challenges that have to be overcome in order to perform automated testing for Scratch.
% These challenges include Scratch's code system, which runs parallel scripts, its IO, which is not trivial to automate for testing,
% as well as some common traits in Scratch programs, which can make automated testing more difficult.

\mnote{Good line for abstract?}
In this work we took on the issue of automated assessment of Scratch programs.
There exist multiple previous approaches that tackled this problem.
We examined Hairball~\cite{hairball}, which analyzes Scratch programs by performing static analysis on them,
as well as ITCH~\cite{itch}, which transforms Scratch programs to automate \texttt{ask} and \texttt{say} blocks.
We described various general challenges that have to be overcome in order to perform automated testing for Scratch.
These challenges include Scratch's code system, which runs parallel scripts, its IO, which is not trivial to automate for testing,
as well as some common traits in Scratch programs, which can make automated testing difficult.
\parspace

As the main contribution of this work, we introduced Whisker, a utility that automates Scratch 3.0's IO,
and described a way to perform runtime testing on Scratch programs by using this automation.
Whisker allows tests to programmatically simulate user input on Scratch programs
and to obtain information about the sprites of the program during the execution of the program under test.
It also offers additional features like measuring statement coverage for Scratch.
% After describing Whisker's main functionality,
% we investigated some challenges of this approach, that we came across.
% Aside from general testing challenges,
% we found that addressing sprites and variables, missing initialization, and time-displaced events
% can pose challenges for automated testing with this approach.
\parspace

We introduced a testing procedure that works by defining constraints that the program must hold.
% Whisker implements these constraints by performing assertions each time Scratch renders a new frame.
Whisker checks these constraints in the background while the program is executed,
which makes it possible to test with various sources of inputs, for example with randomly generated input.
% We described this procedure and compared it to a more traditional testing process.
For this purpose, we also implemented an automated input generation algorithm in Whisker,
which detects what inputs the program can react to, and randomly performs these inputs on the program.
\parspace

We evaluated Whisker in three separate experiments.
In the first experiment, we ran multiple test suites on a set of student-written Scratch programs.
We learned that our testing approach is able to consistently produce test results that closely match the results of manual assessment.
In the second experiment we tested the automated input algorithm by measuring its achieved statement coverage
on a set of different programs.
We observed that it is able to cover most of usual Scratch programs' code,
and produces much higher coverage compared to running the program without any simulated inputs.
Finally, we investigated if the additional computations done by Whisker and the test code would interfere with the Scratch program
in our test executions by slowing it down, and discovered that this is not the case.
\parspace

% We will continue development on Whisker in order to facilitate the process of writing tests as well as the process of executing tests.
% Helper methods may make it possible to write shorter and simpler test code.
% And a standalone user interface may automate test execution and allow users to perform test executions in parallel.%
% \parspace

In conclusion, we find automating Scratch's IO to be a viable way to perform functional testing on Scratch programs,
and to automatically score them.
Therefore, we believe that automated testing may aid in the assessment of Scratch assignments in the future,
both for grading purposes and for students or independent learners to check their own solutions.

%    Conclude and sum up the results.
%    Provide a conclusion to the research questions.

\chapter{Future Work}
\label{cha:future_work}

% What does this work already present?
Although Whisker is pretty usable in its current form,
there are still many opportunities for improvement:
\parspace

\textbf{Automated input.}
While Whisker's automated input algorithm works quite well, it is still very simple.
One could use more elaborate static analysis or an evolutionary algorithm to find better fitting inputs.
For example, the optimal duration of a key press and its probability may be determined through static analysis.
At the same time, the algorithm may construct correct answers to \texttt{ask} blocks at run time.
Figure~\ref{fig:generated_ask_answer} shows an ask-answer configuration from one of Code Club's sample solutions,
for which answers may be generated.

\begin{figure}[htpb]
    \centering
    \includegraphics[width=0.4\textwidth]{scratch-ask-answer}
    \caption{Ask-answer configuration with a generated question and answer}
    \label{fig:generated_ask_answer}
\end{figure}


\textbf{Simplify tests with helper methods.}
Tests with Whisker can quickly become quite long and complex.
Therefore, Whisker should should provide more features that help simplify writing tests.
One task, which currently requires a large amount of code,
is checking temporal relationships between two events,
for example "one second after some sprite touches a border, some variable must be increased".
In the future, testing may be simplified greatly by providing methods to handle cases like this.
\parspace

\textbf{Support for audio and Scratch extensions.}
Whisker could be extended to support sounds and Scratch extensions.
Scratch has a small number of extensions, which add blocks with various functionality.
For example, the "Pen" extension allows to freely draw on the stage by controlling a pen through the program,
and the "Video Sensing" extensions allows to detect movement with a web cam.
Adding support for audio as well as extensions in the future would make Whisker applicable to a wider range of Scratch projects.
\parspace

\textbf{User interface.}
Another thing, that can be expanded upon, is Whisker's user interface.
Currently Whisker only has a web GUI, which is accessed through a web browser.
This interface can be used to test projects in batch, but it still requires manual user interaction to select programs and tests,
and to save the test report once the test execution finished.
In the future, a standalone Electron~\cite{electron} application could solve these problems by allowing Whisker to directly load projects and tests,
and to directly save test reports.
It would also make it possible to run tests in parallel on a cluster.
\parspace

\textbf{Seed randomness.}



\chapter{Appendix}
\label{cha:appendix}


\section{Inconsistencies Matrices}
\label{sec:inconsistencies_matrices}

\begin{table}
    \setlength{\tabcolsep}{0.2em}
    \tiny
    \begin{tabular}{l|rrrrrrrrrrrrrrrrrrrrrrrrrrrr}
        \toprule
        & 01 & 02 & 03 & 04 & 05 & 06 & 07 & 08 & 09 & 10 & 11 & 12 & 13 & 14 & 15 & 16 & 17 & 18 & 19 & 20 & 21 & 22 & 23 & 24 & 25 & 26 & 27 & 28 \\
        \midrule
        K6\_S02 & 0 & 0 & 0 & 0 & 0 & 0 & 0 & 0 & 0 & 0 & 0 & 0 & 0 & \textbf{1} & \textbf{1} & 0 & 0 & 0 & 0 & 0 & 0 & 0 & \textbf{1} & 0 & 0 & 0 & 0 & 0 \\
        K6\_S03 & 0 & 0 & 0 & 0 & 0 & 0 & 0 & \textbf{1} & 0 & 0 & 0 & 0 & \textbf{1} & \textbf{1} & \textbf{1} & 0 & \textbf{1} & 0 & 0 & 0 & 0 & 0 & 0 & 0 & 0 & 0 & 0 & 0 \\
        K6\_S05 & 0 & 0 & 0 & 0 & 0 & 0 & 0 & 0 & 0 & 0 & 0 & 0 & 0 & 0 & 0 & 0 & 0 & 0 & 0 & 0 & 0 & 0 & 0 & 0 & 0 & 0 & 0 & 0 \\
        K6\_S10 & 0 & 0 & 0 & 0 & 0 & 0 & 0 & 0 & 0 & 0 & 0 & 0 & \textbf{1} & \textbf{1} & 0 & 0 & 0 & 0 & 0 & 0 & 0 & 0 & \textbf{1} & 0 & 0 & 0 & 0 & 0 \\
        K6\_S11 & 0 & 0 & 0 & 0 & 0 & 0 & 0 & 0 & 0 & 0 & 0 & 0 & 0 & 0 & 0 & 0 & 0 & 0 & \textbf{1} & 0 & 0 & 0 & 0 & 0 & 0 & 0 & 0 & 0 \\
        K6\_S13 & 0 & 0 & 0 & 0 & 0 & 0 & 0 & 0 & 0 & 0 & 0 & 0 & 0 & 0 & 0 & 0 & 0 & 0 & 0 & 0 & 0 & 0 & 0 & 0 & 0 & 0 & 0 & 0 \\
        K6\_S15 & 0 & 0 & 0 & 0 & 0 & 0 & 0 & 0 & 0 & 0 & 0 & 0 & 0 & 0 & 0 & 0 & 0 & 0 & 0 & 0 & 0 & 0 & 0 & 0 & 0 & 0 & 0 & 0 \\
        K6\_S16 & 0 & 0 & 0 & 0 & 0 & 0 & 0 & 0 & 0 & 0 & 0 & 0 & 0 & 0 & 0 & 0 & 0 & 0 & 0 & 0 & 0 & 0 & 0 & 0 & 0 & 0 & 0 & 0 \\
        K6\_S18 & 0 & 0 & 0 & 0 & 0 & 0 & 0 & 0 & 0 & 0 & 0 & 0 & 0 & 0 & 0 & 0 & 0 & 0 & 0 & 0 & 0 & 0 & 0 & 0 & 0 & 0 & 0 & 0 \\
        K6\_S19 & 0 & 0 & 0 & 0 & 0 & 0 & 0 & 0 & 0 & 0 & 0 & 0 & 0 & 0 & 0 & 0 & 0 & 0 & 0 & 0 & 0 & 0 & \textbf{1} & 0 & 0 & 0 & \textbf{1} & 0 \\
        K6\_S27 & 0 & 0 & 0 & 0 & 0 & 0 & 0 & 0 & 0 & 0 & 0 & 0 & 0 & 0 & 0 & 0 & 0 & 0 & 0 & 0 & 0 & 0 & 0 & 0 & 0 & 0 & 0 & 0 \\
        K6\_S29 & 0 & 0 & 0 & 0 & 0 & 0 & 0 & 0 & 0 & 0 & 0 & 0 & 0 & 0 & 0 & 0 & 0 & 0 & 0 & 0 & 0 & 0 & 0 & 0 & 0 & 0 & 0 & 0 \\
        K6\_S30 & 0 & 0 & 0 & 0 & 0 & 0 & 0 & 0 & 0 & 0 & 0 & 0 & 0 & 0 & 0 & 0 & 0 & 0 & 0 & 0 & 0 & 0 & 0 & 0 & 0 & 0 & 0 & 0 \\
        K6\_S31 & 0 & 0 & 0 & 0 & 0 & 0 & 0 & 0 & 0 & 0 & 0 & 0 & 0 & 0 & 0 & 0 & 0 & 0 & 0 & 0 & 0 & 0 & 0 & 0 & 0 & 0 & 0 & 0 \\
        K6\_S33 & 0 & 0 & 0 & 0 & 0 & 0 & 0 & \textbf{1} & \textbf{1} & 0 & 0 & 0 & 0 & 0 & 0 & 0 & 0 & 0 & 0 & 0 & 0 & 0 & 0 & 0 & 0 & 0 & 0 & 0 \\
        K7\_S02 & 0 & 0 & 0 & 0 & 0 & 0 & 0 & 0 & 0 & 0 & 0 & 0 & 0 & 0 & 0 & 0 & 0 & 0 & 0 & 0 & \textbf{1} & 0 & 0 & 0 & 0 & 0 & 0 & 0 \\
        K7\_S03 & 0 & 0 & 0 & 0 & 0 & 0 & 0 & 0 & 0 & 0 & 0 & 0 & \textbf{1} & \textbf{1} & 0 & 0 & \textbf{1} & 0 & 0 & 0 & 0 & 0 & 0 & 0 & 0 & 0 & 0 & 0 \\
        K7\_S04 & 0 & 0 & 0 & 0 & 0 & 0 & 0 & 0 & 0 & 0 & 0 & 0 & 0 & 0 & 0 & 0 & 0 & 0 & 0 & 0 & 0 & 0 & 0 & 0 & 0 & 0 & 0 & 0 \\
        K7\_S05 & 0 & 0 & 0 & 0 & 0 & 0 & 0 & 0 & 0 & 0 & 0 & 0 & 0 & 0 & 0 & 0 & 0 & 0 & 0 & 0 & 0 & 0 & 0 & 0 & 0 & 0 & 0 & 0 \\
        K7\_S06 & 0 & 0 & 0 & 0 & 0 & 0 & 0 & 0 & 0 & 0 & 0 & 0 & 0 & 0 & 0 & 0 & 0 & 0 & 0 & 0 & 0 & 0 & 0 & 0 & 0 & 0 & 0 & 0 \\
        K7\_S07 & 0 & 0 & 0 & 0 & 0 & 0 & 0 & 0 & 0 & 0 & 0 & 0 & \textbf{1} & 0 & \textbf{1} & 0 & \textbf{1} & 0 & 0 & 0 & 0 & 0 & \textbf{1} & 0 & 0 & 0 & 0 & 0 \\
        K7\_S08 & 0 & 0 & 0 & 0 & 0 & 0 & 0 & 0 & 0 & 0 & 0 & 0 & 0 & 0 & 0 & 0 & 0 & 0 & 0 & 0 & 0 & 0 & 0 & 0 & 0 & 0 & 0 & 0 \\
        K7\_S10 & 0 & 0 & 0 & 0 & 0 & 0 & 0 & 0 & 0 & 0 & 0 & 0 & 0 & 0 & 0 & 0 & 0 & 0 & 0 & 0 & \textbf{1} & 0 & 0 & 0 & 0 & 0 & 0 & 0 \\
        K7\_S11 & 0 & 0 & 0 & 0 & 0 & 0 & 0 & 0 & 0 & 0 & 0 & 0 & 0 & 0 & 0 & 0 & 0 & 0 & 0 & 0 & 0 & 0 & 0 & 0 & 0 & 0 & 0 & 0 \\
        K7\_S12 & 0 & 0 & 0 & 0 & 0 & 0 & 0 & 0 & 0 & 0 & 0 & 0 & 0 & \textbf{1} & 0 & 0 & 0 & 0 & 0 & 0 & 0 & 0 & 0 & 0 & 0 & 0 & 0 & 0 \\
        K7\_S14 & 0 & 0 & 0 & 0 & 0 & 0 & 0 & 0 & 0 & 0 & 0 & 0 & 0 & 0 & 0 & 0 & 0 & 0 & 0 & 0 & 0 & 0 & 0 & 0 & 0 & 0 & 0 & 0 \\
        K7\_S15 & 0 & 0 & 0 & 0 & 0 & 0 & 0 & 0 & 0 & 0 & 0 & 0 & 0 & 0 & 0 & 0 & 0 & 0 & 0 & 0 & 0 & 0 & 0 & \textbf{1} & 0 & 0 & 0 & 0 \\
        K7\_S16 & 0 & 0 & 0 & \textbf{1} & \textbf{1} & 0 & 0 & 0 & 0 & 0 & 0 & 0 & 0 & 0 & 0 & 0 & 0 & 0 & 0 & 0 & 0 & 0 & 0 & 0 & 0 & 0 & 0 & 0 \\
        K7\_S17 & 0 & 0 & 0 & 0 & 0 & 0 & 0 & 0 & 0 & 0 & 0 & 0 & 0 & 0 & 0 & 0 & 0 & 0 & \textbf{1} & 0 & \textbf{1} & 0 & 0 & \textbf{1} & 0 & 0 & 0 & 0 \\
        K7\_S19 & 0 & 0 & 0 & 0 & 0 & 0 & 0 & 0 & 0 & 0 & 0 & 0 & 0 & 0 & 0 & 0 & 0 & 0 & 0 & 0 & 0 & 0 & \textbf{1} & 0 & 0 & 0 & 0 & 0 \\
        K7\_S20 & 0 & 0 & 0 & 0 & 0 & 0 & 0 & 0 & 0 & 0 & 0 & 0 & 0 & 0 & 0 & 0 & 0 & 0 & 0 & 0 & 0 & \textbf{1} & 0 & 0 & 0 & 0 & 0 & 0 \\
        K7\_S26 & 0 & 0 & 0 & 0 & 0 & 0 & 0 & 0 & 0 & 0 & 0 & 0 & 0 & 0 & 0 & 0 & 0 & 0 & 0 & 0 & 0 & 0 & 0 & \textbf{1} & 0 & 0 & 0 & 0 \\
        \bottomrule
    \end{tabular}
    \caption{idk}
    \label{tab:inconsistencies_matrix_normal}
    \setlength{\tabcolsep}{\defaulttabcolsep}
    \mnote{TODO: update}
\end{table}

\begin{table}
    \setlength{\tabcolsep}{0.2em}
    \tiny
    \begin{tabular}{l|rrrrrrrrrrrrrrrrrrrrrrrrrrrr}
        \toprule
        & 01 & 02 & 03 & 04 & 05 & 06 & 07 & 08 & 09 & 10 & 11 & 12 & 13 & 14 & 15 & 16 & 17 & 18 & 19 & 20 & 21 & 22 & 23 & 24 & 25 & 26 & 27 & 28 \\
        \midrule
        K6\_S02 & 0 & 0 & 0 & 0 & 0 & 0 & 0 & 0 & 0 & 0 & 0 & 0 & 0 & \textbf{1} & \textbf{1} & 0 & 0 & 0 & 0 & 0 & 0 & 0 & \textbf{1} & 0 & 0 & 0 & 0 & 0 \\
        K6\_S03 & 0 & 0 & 0 & 0 & 0 & 0 & 0 & \textbf{1} & 0 & 0 & 0 & 0 & \textbf{1} & \textbf{1} & \textbf{1} & 0 & \textbf{1} & 0 & 0 & 0 & 0 & 0 & 0 & 0 & 0 & 0 & 0 & 0 \\
        K6\_S05 & 0 & 0 & 0 & 0 & 0 & 0 & 0 & 0 & 0 & 0 & 0 & 0 & 0 & 0 & 0 & 0 & 0 & 0 & 0 & 0 & 0 & 0 & 0 & 0 & 0 & 0 & 0 & 0 \\
        K6\_S10 & 0 & 0 & 0 & 0 & 0 & 0 & 0 & 0 & 0 & 0 & 0 & 0 & \textbf{1} & \textbf{1} & 0 & 0 & 0 & 0 & 0 & 0 & 0 & 0 & \textbf{1} & 0 & 0 & 0 & 0 & 0 \\
        K6\_S11 & 0 & 0 & 0 & 0 & 0 & 0 & 0 & 0 & 0 & 0 & 0 & 0 & 0 & 0 & 0 & 0 & 0 & 0 & \textbf{1} & 0 & 0 & 0 & 0 & 0 & 0 & 0 & 0 & 0 \\
        K6\_S13 & 0 & 0 & 0 & 0 & 0 & 0 & 0 & 0 & 0 & 0 & 0 & 0 & 0 & 0 & 0 & 0 & 0 & 0 & 0 & 0 & 0 & 0 & 0 & 0 & 0 & 0 & 0 & 0 \\
        K6\_S15 & 0 & 0 & 0 & 0 & 0 & 0 & 0 & 0 & 0 & 0 & 0 & 0 & 0 & 0 & 0 & 0 & 0 & 0 & 0 & 0 & 0 & 0 & 0 & 0 & 0 & 0 & 0 & 0 \\
        K6\_S16 & 0 & 0 & 0 & 0 & 0 & 0 & 0 & 0 & 0 & 0 & 0 & 0 & 0 & 0 & 0 & 0 & 0 & 0 & 0 & 0 & 0 & 0 & 0 & 0 & 0 & 0 & 0 & 0 \\
        K6\_S18 & 0 & 0 & 0 & 0 & 0 & 0 & 0 & 0 & 0 & 0 & 0 & 0 & 0 & 0 & 0 & 0 & 0 & 0 & 0 & 0 & 0 & 0 & 0 & 0 & 0 & 0 & 0 & 0 \\
        K6\_S19 & 0 & 0 & 0 & 0 & 0 & 0 & 0 & 0 & 0 & 0 & 0 & 0 & 0 & 0 & 0 & 0 & 0 & 0 & 0 & 0 & 0 & 0 & \textbf{1} & 0 & 0 & 0 & \textbf{1} & 0 \\
        K6\_S27 & 0 & 0 & 0 & 0 & 0 & 0 & 0 & 0 & 0 & 0 & 0 & 0 & 0 & 0 & 0 & 0 & 0 & 0 & 0 & 0 & 0 & 0 & 0 & 0 & 0 & 0 & 0 & 0 \\
        K6\_S29 & 0 & 0 & 0 & 0 & 0 & 0 & 0 & 0 & 0 & 0 & 0 & 0 & 0 & 0 & 0 & 0 & 0 & 0 & 0 & 0 & 0 & 0 & 0 & 0 & 0 & 0 & 0 & 0 \\
        K6\_S30 & 0 & 0 & 0 & 0 & 0 & 0 & 0 & 0 & 0 & 0 & 0 & 0 & 0 & 0 & 0 & 0 & 0 & 0 & 0 & 0 & 0 & 0 & 0 & 0 & 0 & 0 & 0 & 0 \\
        K6\_S31 & 0 & 0 & 0 & 0 & 0 & 0 & 0 & 0 & 0 & 0 & 0 & 0 & 0 & 0 & 0 & 0 & 0 & 0 & 0 & 0 & 0 & 0 & 0 & 0 & 0 & 0 & 0 & 0 \\
        K6\_S33 & 0 & 0 & 0 & 0 & 0 & 0 & 0 & \textbf{1} & \textbf{1} & 0 & 0 & 0 & 0 & 0 & 0 & 0 & 0 & 0 & 0 & 0 & 0 & 0 & 0 & 0 & 0 & 0 & 0 & 0 \\
        K7\_S02 & 0 & 0 & 0 & 0 & 0 & 0 & 0 & 0 & 0 & 0 & 0 & 0 & 0 & 0 & 0 & 0 & 0 & 0 & 0 & 0 & \textbf{1} & 0 & 0 & 0 & 0 & 0 & 0 & 0 \\
        K7\_S03 & 0 & 0 & 0 & 0 & 0 & 0 & 0 & 0 & 0 & 0 & 0 & 0 & \textbf{1} & \textbf{1} & 0 & 0 & \textbf{1} & 0 & 0 & 0 & 0 & 0 & 0 & 0 & 0 & 0 & 0 & 0 \\
        K7\_S04 & 0 & 0 & 0 & 0 & 0 & 0 & 0 & 0 & 0 & 0 & 0 & 0 & 0 & 0 & 0 & 0 & 0 & 0 & 0 & 0 & 0 & 0 & 0 & 0 & 0 & 0 & 0 & 0 \\
        K7\_S05 & 0 & 0 & 0 & 0 & 0 & 0 & 0 & 0 & 0 & 0 & 0 & 0 & 0 & 0 & 0 & 0 & 0 & 0 & 0 & 0 & 0 & 0 & 0 & 0 & 0 & 0 & 0 & 0 \\
        K7\_S06 & 0 & 0 & 0 & 0 & 0 & 0 & 0 & 0 & 0 & 0 & 0 & 0 & 0 & 0 & 0 & 0 & 0 & 0 & 0 & 0 & 0 & 0 & 0 & 0 & 0 & 0 & 0 & 0 \\
        K7\_S07 & 0 & 0 & 0 & 0 & 0 & 0 & 0 & 0 & 0 & 0 & 0 & 0 & \textbf{1} & 0 & \textbf{1} & 0 & \textbf{1} & 0 & 0 & 0 & 0 & 0 & \textbf{1} & 0 & 0 & 0 & 0 & 0 \\
        K7\_S08 & 0 & 0 & 0 & 0 & 0 & 0 & 0 & 0 & 0 & 0 & 0 & 0 & 0 & 0 & 0 & 0 & 0 & 0 & 0 & 0 & 0 & 0 & 0 & 0 & 0 & 0 & 0 & 0 \\
        K7\_S10 & 0 & 0 & 0 & 0 & 0 & 0 & 0 & 0 & 0 & 0 & 0 & 0 & 0 & 0 & 0 & 0 & 0 & 0 & 0 & 0 & \textbf{1} & 0 & 0 & 0 & 0 & 0 & 0 & 0 \\
        K7\_S11 & 0 & 0 & 0 & 0 & 0 & 0 & 0 & 0 & 0 & 0 & 0 & 0 & 0 & 0 & 0 & 0 & 0 & 0 & 0 & 0 & 0 & 0 & 0 & 0 & 0 & 0 & 0 & 0 \\
        K7\_S12 & 0 & 0 & 0 & 0 & 0 & 0 & 0 & 0 & 0 & 0 & 0 & 0 & 0 & \textbf{1} & 0 & 0 & 0 & 0 & 0 & 0 & 0 & 0 & 0 & 0 & 0 & 0 & 0 & 0 \\
        K7\_S14 & 0 & 0 & 0 & 0 & 0 & 0 & 0 & 0 & 0 & 0 & 0 & 0 & 0 & 0 & 0 & 0 & 0 & 0 & 0 & 0 & 0 & 0 & 0 & 0 & 0 & 0 & 0 & 0 \\
        K7\_S15 & 0 & 0 & 0 & 0 & 0 & 0 & 0 & 0 & 0 & 0 & 0 & 0 & 0 & 0 & 0 & 0 & 0 & 0 & 0 & 0 & 0 & 0 & 0 & \textbf{1} & 0 & 0 & 0 & 0 \\
        K7\_S16 & 0 & 0 & 0 & \textbf{1} & \textbf{1} & 0 & 0 & 0 & 0 & 0 & 0 & 0 & 0 & 0 & 0 & 0 & 0 & 0 & 0 & 0 & 0 & 0 & 0 & 0 & 0 & 0 & 0 & 0 \\
        K7\_S17 & 0 & 0 & 0 & 0 & 0 & 0 & 0 & 0 & 0 & 0 & 0 & 0 & 0 & 0 & 0 & 0 & 0 & 0 & \textbf{1} & 0 & \textbf{1} & 0 & 0 & \textbf{1} & 0 & 0 & 0 & 0 \\
        K7\_S19 & 0 & 0 & 0 & 0 & 0 & 0 & 0 & 0 & 0 & 0 & 0 & 0 & 0 & 0 & 0 & 0 & 0 & 0 & 0 & 0 & 0 & 0 & \textbf{1} & 0 & 0 & 0 & 0 & 0 \\
        K7\_S20 & 0 & 0 & 0 & 0 & 0 & 0 & 0 & 0 & 0 & 0 & 0 & 0 & 0 & 0 & 0 & 0 & 0 & 0 & 0 & 0 & 0 & \textbf{1} & 0 & 0 & 0 & 0 & 0 & 0 \\
        K7\_S26 & 0 & 0 & 0 & 0 & 0 & 0 & 0 & 0 & 0 & 0 & 0 & 0 & 0 & 0 & 0 & 0 & 0 & 0 & 0 & 0 & 0 & 0 & 0 & \textbf{1} & 0 & 0 & 0 & 0 \\
        \bottomrule
    \end{tabular}
    \caption{idk}
    \label{tab:inconsistencies_matrix_constraint}
    \setlength{\tabcolsep}{\defaulttabcolsep}
    \mnote{TODO: update}
\end{table}












































































TODO: REMOVE THIS
\label{TODO}
TODO: REMOVE THIS

% Bibliography
\clearpage
%\bibliographystyle{apalike}
\bibliographystyle{acm}
\bibliography{references}

\end{document}
