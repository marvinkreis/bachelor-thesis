% REMEMBER: Write the thesis from the view of the reader. How would I like to READ the thesis?
% WHY -> WHAT -> HOW structure

% TODO: Explain Whisker GUI (or in approach?)
    % Whisker can theoretically be run on any instance of the Scratch VM
    % We implement a simple GUI which also shows loaded tests and their output
% TODO: Explain test output, TAP13 report (or in approach?)
% TODO: Explain general design
    % In theory, Whisker can be used with any instance of the Scratch virtual machine.
    % In order to make testing more convenient, we developed a small web interface with its own testing framework, which also

% Environment:
% ES6 + webpack
% ...

% General Design
%   - interact with the VM
%   - doesn't change the VM
%   - works with any working Scratch 3.0 VM instance
%   - works with any JS testing framework
%   - Scratch needs renderer
%       - Web GUI + Electron planned + maybe browser addon?

% Step Loop: explain steps, explain order of steps
% Outputs: Sprites and variables
% Inputs
% Constraints


% Scratch 3.0 implementation details


\chapter{Implementation}

TODO: keyword "automation"

- This section shows a implementation of a testing utility for Scratch
- Uses the approach described above
- Implemented in JavaScript (ES6) for compatibility with Scratch 3.0, which is implemented in JavaScript

\section {Scratch 3.0 Implementation Details}

Scratch's interpreter sequentializes the execution.
This is necessary for the single-threaded JavaScript environment Scratch is run in.
This means that no race-conditions can occur on a language level.

\section{General Design}

- TODO explain Scratch's design?
- TODO explain why it's possible to execute code before and after Scratch's step without interfering with the Scratch program

- Tests are written in JavaScript
- Usually one test is a function of a .js file

- Does not change the virtual machine in any way
    $\rightarrow$ designed so it can be used with any instance of the Scratch virtual machine
    $\rightarrow$ same version of Whisker can be used with different versions of the Scratch virtual machine
    $\rightarrow$ even something like a browser addon for the original Scratch page would be possible

- Whisker is designed to be a layer between test code and the scratch virtual machine
- Allows to interact with the VM in a test-friendly way
- The main class VM Wrapper and its components make up a wrapper around the scratch virtual machine,
  which offers extra functionality for testing
- Test Driver offers a user-friendly interface between the test code and the VM Wrapper
- Test uses Test Driver to simulate input, get information about sprites

- Test Driver could be acquired through a helper method or passed to the test method
- Up to the testing framework to provide the test driver
- The examples in this chapter will refer to the test driver with the variable \texttt{t}

\begin{listing}[ht]
    \centering
    \begin{minipage}[t]{.45\textwidth}
        \begin{javascriptcode}
            async function test (t) {
                ...
            }
        \end{javascriptcode}
        \vspace{-\bigskipamount}
        Getting the test driver as a parameter to the test function
    \end{minipage}
    ~
    \begin{minipage}[t]{.45\textwidth}
        \begin{javascriptcode}
            async function test () {
                const t = acquireTestDriver();
            }
        \end{javascriptcode}
        \vspace{-\bigskipamount}
        Getting the test driver through some helper method
    \end{minipage}
    \caption{Examples of how to acquire the test driver}
    \label{fig:examples_of_how_to_acquire_the_test_driver}
\end{listing}

=== Limitations
- Scratch depends on the renderer
    - Some functionality of the Scratch virtual machine depends on the renderer
    - Headless tests are impossible without restricting the Scratch program to a subset of available blocks
- Therefore Whisker assumes the VM is always run with a renderer in place

\begin{figure}[ht]
    \centering
    \tikzset{>=latex,
             arrow/.style={-{Latex[length=1.5mm, width=1.5mm]}},
             label/.style={draw=none, text width=5.3cm, minimum height=0.5cm, text centered},
               box/.style={draw,      text width=2.5cm, minimum height=0.7cm, text centered, rounded corners},
                 h/.style={fill=blue!10}}

    \begin{tikzpicture}
        \node[box]   at ( 0.0,  3.0) (testcode)      {Test Code};
        \node[box]   at ( 0.0,  1.5) (testdriver)    {Test Driver};
        \node[label] at ( 0.0,  0.0) (vmwrapper)     {VM Wrapper};
        \node[box]   at (-1.4, -0.7) (sprites)       {Sprites};
        \node[box]   at (-1.4, -1.6) (inputs)        {Inputs};
        \node[box]   at ( 1.5, -0.7) (callbacks)     {Callbacks};
        \node[box]   at ( 1.5, -1.6) (constraints)   {Constraints};
        \node[box]   at (-2.0, -3.2) (scratchvm)     {Scratch VM};
        \node[box]   at ( 2.2, -3.2) (scratchrender) {Renderer};

        \begin{scope}[on background layer]
            \node[draw, h, rounded corners, fit=(vmwrapper)(sprites)(inputs)(callbacks)(constraints)] (container) {};
        \end{scope}

        \foreach \pp/\pf/\pt in {--/testcode/testdriver,
                                 --/testdriver/container,
                                 --/container/scratchvm,
                                 --/container/scratchrender,
                                 --/scratchvm/scratchrender}
            \draw[shorten >= 2pt, arrow] (\pf) \pp (\pt);

        % \draw[shorten >= 2pt, rounded corners, dashed, ->]
        %        (constraints)
        %     -- ( 3.5, -1.6)
        %     -- ( 3.5,  3.0)
        %     -- (testcode);
        % \draw[dashed, -] (callbacks) -- ( 3.5, -0.7);
    \end{tikzpicture}

    \caption{Components of Whisker, TODO: put testing framework in the diagram}
    \label{fig:components_of_whisker}
\end{figure}

\section{The Step Loop}

- The core of the Scratch virtual machine is a step-function, which is called at a constant interval (using JavaScript's \texttt{setInterval()})
- Interval of 30 times / second for Scratch 2.0, 60 times / second for Scratch 3.0
- The function executes the program until a time limit is reached and then redraws the scene
- If some visual change occurs in the project, the program execution is stopped earlier and the scene is rendered

\begin{listing}[ht]
    \centering
    \begin{javascriptcode}
        STEP_TIME = 1000 / STEPS_PER_SECOND;
        WORK_TIME = 0.75 * STEP_TIME;

        while (running &&
               timeElapsed < WORK_TIME &&
               !redrawRequested) {
            for (thread of threads) {
                stepThread(thread);
            }
        }

        renderer.draw();
    \end{javascriptcode}
    \vspace{-\bigskipamount}
    \caption{Simplified Scratch Step Procedure}
    \label{fig:simplified_scratch_step_procedure}
\end{listing}

- Instead of executing the step function via interval, Whisker executes its own step loop, which calls Scratch's step function
- Before and after the step of the Scratch program, test code is run, registered inputs are performed and sprite objects are updated

- This should either not affect the Scratch program, or only affect it minimally
    - Renderer might not need to use the entire allocated rendering time
    - If something changes in a Scratch program, usually a sprite moves $\rightarrow$ Scratch will only use a fraction of the entire allocated work time in most cases
    - Scratch uses real time to track wait times $\rightarrow$ not affected much by a step that takes longer that normally

=== Scratch programs depend on real time
- Possible problem: additional computations could cause the program to run slower
- Scratch blocks that involve timings, like "wait" or "say for secs" use real time to delay execution
$\rightarrow$ If the additional computations take too long, they could influence the program

\begin{figure}[ht]
    \centering
    \tikzset{>=latex,
           arrow/.style={draw, -{Latex[length=1.5mm, width=1.5mm]}},
             box/.style={draw, text width=4.3cm, minimum height=0.7cm, text centered, rounded corners},
             num/.style={draw, circle, inner sep=0.6mm, text centered},
               h/.style={fill=blue!10}}

    \begin{tikzpicture}
        \node[box]    at ( 0.2,  5.0) (callbacksbefore) {Call Callbacks (before)};
        \node[box]    at ( 0.2,  4.0) (inputs)          {Perform Inputs};
        \node[box]    at ( 0.2,  3.0) (sprites)         {Update Sprites};
        \node[box, h] at ( 0.2,  2.0) (step)            {Step Scratch Program};
        \node[box]    at ( 0.2,  1.0) (callbacksafter)  {Call Callbacks (after)};
        \node[box]    at ( 0.2,  0.0) (constraints)     {Check Constraints};

        \node[text width=2cm] at (-4.5, 2.5) (wait)  {Wait until next step is due};

        \node[num] at (-2.6,  5.0) (one)   {1};
        \node[num] at (-2.6,  4.0) (two)   {2};
        \node[num] at (-2.6,  3.0) (three) {3};
        \node[num] at (-2.6,  2.0) (four)  {4};
        \node[num] at (-2.6,  1.0) (five)  {5};
        \node[num] at (-2.6,  0.0) (six)   {6};

        \draw[arrow]
               (callbacksbefore)
            -- (inputs)
            -- (sprites)
            -- (step)
            -- (callbacksafter)
            -- (constraints)
            -- ( 0.2, -1.5);

        \draw[shorten >= 2pt, rounded corners, dashed, arrow]
               ( 0.2, -1.0)
            -- (-3.4, -1.0)
            -- (-3.4,  6.0)
            -- ( 0.2,  6.0)
            -- (callbacksbefore);
    \end{tikzpicture}

    \caption{Whisker Step Procedure}
    \label{fig:whisker_step_procedure}
\end{figure}

% \section{Basic Testing Functionality}
%
% === WM Wrapper
% - Control the execution of the scratch program
% - Run the program until a certain amount of time has passed or a condition has been met
% - Get the time elapsed since the start of the test or the start of the last run
% - Cancel a run
% - Uses JavaScript's Promise API to wait until a run is finished
%
% \begin{listing}[ht]
%     \centering
%     \begin{javascriptcode}
%         await t.runForTime(500);
%         await t.runUntil(() => !t.projectRunning(), 1000);
%         t.assert.ok(t.getTotalTimeElapsed() < 1000);
%     \end{javascriptcode}
%     \vspace{-\bigskipamount}
%     \caption{Example code for the VM Wrapper}
%     \label{fig:example_code_for_the_vm_wrapper}
% \end{listing}
%
% === Sprites
% - Sprite is not the same as sprite in Scratch VM
%     - Explain distinguishes between sprites and targets / rendered targets
%     - Sprite contains the blocks, graphics (costumes), etc.
%     - Target / rendered target is an instance of the sprite
%     - Whisker sees every rendered target as a sprite
%     - The original target of a Scratch sprite as well as its clones are each an instance of a ''sprites''
%     - TODO: explain clones here
%
% - Sprites work by wrapping around the original
% - If some getter of the sprite is called, the actual value is retrieved from the original target
% - Most properties are implemented as JavaScript getters $\rightarrow$ look like properties of Whisker's sprite object
%
% - Information about sprites and variables
% - Gives the information that the test uses
% - Does not allow to manipulate sprites and variables
% - Contains ''old'' value for every fitting property
%     - Saves the value from the last step
%     - Useful for constraints (see later)
%     - Initialized with the present value
% - Sprites are only tracked once they are retrieved via one of the getter method
% - Helps, for example, with programs that spawn a lot of clones (could pose a performance problem otherwise)
%
%
% \begin{listing}[ht]
%     \centering
%     \begin{javascriptcode}
%         const sprite = t.getSprite('Sprite1');
%         const variable = sprite.getVariable('Variable1');
%         const sprites = t.getSprites(s => s.x > 100);
%
%         t.assert.equal(sprite.x, 100);
%         t.assert.equal(sprite.old.x, 100);
%         t.assert.equal(variable.value, 5);
%     \end{javascriptcode}
%     \vspace{-\bigskipamount}
%     \caption{Example code for Sprites}
%     \label{fig:example_code_for_sprites}
% \end{listing}
%
% === Inputs
% - INPUTS ARE SIMULATED ON THE VM, NO ACTIONS ARE SIMULATED ON THE OS LEVEL OR ANYTHING
%
% - Simulate inputs on the program
% - Can be registered to be called after a certain amount of time or be executed immediately
% - Registering a Input with 0 delay is different from executing it immediately
% === Kinds of Input
% - At the moment: only mouse and keyboard input
% - Keyboard:
%     - Press a key
%     - Release a key
%     - Toggle a key
%     - Press / release a key for a certain duration
% - Mouse (only left mouse button):
%     - move cursor to position
%     - move cursor to sprite (+offset)
%     - press mouse button
%     - release mouse button
%     - toggle mouse button
%     - Press / release mouse button for a certain duration
%
% \begin{listing}[ht]
%     \centering
%     \begin{javascriptcode}
%         t.inputImmediate({
%             device: 'keyboard',
%             key: 'right arrow',
%             isDown: true
%         });
%
%         const mouseInput = t.addInput(1000, {
%             device: 'mouse',
%             x: 100,
%             y: 0,
%             isDown: true,
%             duration: 500
%         });
%
%         t.assert.ok(t.isKeyDown('right arrow'));
%         t.removeInput(mouseInput);
%         t.addInput(mouseInput);
%     \end{javascriptcode}
%     \vspace{-\bigskipamount}
%     \caption{Example code for Random Inputs}
%     \label{fig:example_code_for_random_inputs}
% \end{listing}
%
% === Random Inputs
% - Provides a simple way to perform inputs randomly
% - Way of testing the program without deliberately controlling the inputs
% - In a set time interval (at the next step), an input is randomly selected from a pool of registered random inputs
%
% - You  can register inputs for the random pool, or let Whisker choose inputs based on the blocks used in the program
% - Random inputs can be detected through simple static analysis
% $\rightarrow$ Blocks that take inputs and their options are analyzed
%
% - TODO table of detected blocks and resulting inputs ?
%
% \begin{listing}[ht]
%     \centering
%     \begin{javascriptcode}
%         t.setRandomInputInterval(150);
%         t.registerRandomInputs([
%             {
%                 device: 'keyboard',
%                 key: 'left arrow',
%                 duration: [50, 100]
%             },
%             {
%                 device: 'keyboard',
%                 key: 'right arrow',
%                 duration: [50, 100]
%             }
%         ]);
%         t.detectRandomInputs();
%     \end{javascriptcode}
%     \vspace{-\bigskipamount}
%     \caption{Example code for Inputs}
%     \label{fig:example_code_for_inputs}
% \end{listing}
%
% === Callbacks
% - Register callbacks that are called after every step
% - Can be registered to run at two positions in the step cycle (see diagram later)
% - Purpose:
%     - Information Tracking:
%         - Track events
%         - e.g how many times a sprite touches some other sprite
%               or how long a sprite is invisible, etc.
%     - Inputs:
%         - Allows performing conditional inputs
%         - Control the program like a player would
%         - e.g. follow a sprite with the mouse cursor
%
% \begin{listing}[ht]
%     \centering
%     \begin{javascriptcode}
%         const sprite = t.getSprite('Sprite1');
%
%         const callback = t.addCallback(() => {
%             t.inputImmediate({
%                 device: 'mouse',
%                 x: sprite.x,
%                 y: sprite.y
%             });
%         });
%
%         await t.runForTime(1000);
%
%         callback.disable();
%         callback.enable();
%     \end{javascriptcode}
%     \vspace{-\bigskipamount}
%     \caption{Example code for Callbacks}
%     \label{fig:example_code_for_callbacks}
% \end{listing}
%
%
% === Constraints
% - Describe conditions that must always hold
% - Failed constraints can be configured to fail the test, stop the current program run (\texttt{run...()}), or do nothing
% - Can be used to perform checks like "sprite xy is always visible when sprite yz is visible"
% - Implemented as callbacks that execute assertions
%     - Advantages:
%         - Concise syntax
%         - Multiple assertions per constraint possible
%         - Assertion messages can be used to describe the constraint failure
%     - Disadvantages:
%         - Assertion methods have to be efficient, e.g. if assertion message is constructed every time it could be too slow
%         - Need to catch exceptions if constraints should not fail the test
% - Can separate assertions from the program execution
%     - Define constraints for tested properties, then use whatever input (deliberate / random / manual)
%     - Problem: most constraints still hold if the tested property is not implemented at all
%         - e.g. constraint that checks that a sprite never moves left will hold if the sprite doesn't move at all
%
% \begin{listing}[ht]
%     \centering
%     \begin{javascriptcode}
%         const sprite = t.getSprite('Sprite1');
%
%         t.onConstraintFailure('fail');
%
%         const constraint = t.addConstraint(() => {
%             t.assert.ok(sprite.x >= sprite.old.x);
%         });
%
%         await t.runForTime(1000);
%
%         constraint.disable();
%     \end{javascriptcode}
%     \vspace{-\bigskipamount}
%     \caption{Example code for Constraints}
%     \label{fig:example_code_for_constraints}
% \end{listing}

\section{Automated Test Input Generation}

\section{Coverage Measurement}
- Whisker can measure simple statement coverage
- Project wide and for single sprites / the stage

- Only measures blocks that are part of a script / connected to a hat (this includes procedure definitions)
- Other blocks are obviously not reachable

- Useful to detect if a project has been properly captured by a test
    - If coverage is low, there is a problem with the test or project
    - Or the project contains unnecessary code

- Initialized by traversing the scripts and noting each block id of the blocks in a project
- Coverage is simply measured by tracking which block ids are put on top of the stack of the threads in the Scratch virtual machine

\section{Running Tests}
\label{sec:running_tests}
- Whisker comes with an optional testing framework
- Include a sample test report?

=== Seeing test output, interactive tests
- Users will want to see the program's output while it is run
    - to check if the tests run correctly, to check if the program runs correctly
    - Difficult to determine a problem with the project from just textual test reports
    - Tests without showing output are not very useful in such an interactive environment
$\rightarrow$ Has to be able to run in a interactive environment
    - Web GUI, which can be run by any modern browser
    - Allows users to run individual tests on the project and see the program execution

=== Batch Testing of Projects
- Some tests for Scratch projects can take a long time because projects run in real time
    - raises the need to test scratch programs in parallel
    - Scratch depends on the renderer
        - Some functionality of the Scratch virtual machine depends on the renderer
        - Headless tests are impossible without restricting the Scratch program to a subset of available blocks
$\rightarrow$ Web GUI has the option to run tests on multiple projects sequentially, but this might still take a long time depending on the project and the test suite
$\rightarrow$ Electron
    - Running tests in multiple processes could circumvent the problem
    - Electron provides a renderer that can be used to render the Scratch output to
    - Spawns multiple processes which open a window each, one project is tested in each window
