\chapter{Future Work}
\label{cha:future_work}

% What does this work already present?
Although Whisker is pretty usable in its current form,
there are still many opportunities for improvement:
\parspace

\textbf{Automated input.}
While Whisker's automated input algorithm works quite well, it is still very simple.
One could use more elaborate static analysis or an evolutionary algorithm to find better fitting inputs.
For example, the optimal duration of a key press and its probability may be determined through static analysis.
At the same time, the algorithm may construct correct answers to \texttt{ask} blocks at run time.
Figure~\ref{fig:generated_ask_answer} shows an ask-answer configuration from one of Code Club's sample solutions,
for which answers may be generated.

\begin{figure}[htpb]
    \centering
    \includegraphics[width=0.4\textwidth]{scratch-ask-answer}
    \caption{Ask-answer configuration with a generated question and answer}
    \label{fig:generated_ask_answer}
\end{figure}


\textbf{Simplify tests with helper methods.}
Tests with Whisker can quickly become quite long and complex.
Therefore, Whisker should should provide more features that help simplify writing tests.
One task, which currently requires a large amount of code,
is checking temporal relationships between two events,
for example "one second after some sprite touches a border, some variable must be increased".
In the future, testing may be simplified greatly by providing methods to handle cases like this.
\parspace

\textbf{Support for audio and Scratch extensions.}
Whisker could be extended to support sounds and Scratch extensions.
Scratch has a small number of extensions, which add blocks with various functionality.
For example, the "Pen" extension allows to freely draw on the stage by controlling a pen through the program,
and the "Video Sensing" extensions allows to detect movement with a web cam.
Adding support for audio as well as extensions in the future would make Whisker applicable to a wider range of Scratch projects.
\parspace

\textbf{User interface.}
Another thing, that can be expanded upon, is Whisker's user interface.
Currently Whisker only has a web GUI, which is accessed through a web browser.
This interface can be used to test projects in batch, but it still requires manual user interaction to select programs and tests,
and to save the test report once the test execution finished.
In the future, a standalone Electron~\cite{electron} application could solve these problems by allowing Whisker to directly load projects and tests,
and to directly save test reports.
It would also make it possible to run tests in parallel on a cluster.
\parspace

\textbf{Seed randomness.}

