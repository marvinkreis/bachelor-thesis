\chapter{Introduction}

\section{Overview}
=== What is scratch and why does it matter
- Movement to educational, interactive programming languages in introductory courses
- Scratch is one of the widest spread educational programming language
- Its interactive environment lets users easily build animated interactive programs

=== What purpose does automatic testing have in this context
- Student solutions may need to be evaluated
- Currently no automatic solution to this exists
- Manual assessment of scratch programs is very time consuming
- Interactive nature of scratch programs means much interaction with the program is needed to test it
- Every project has to be opened Individually
- Running the can require many key presses and mouse clicks and can take a long time

- Scratch's IO does not allow to automate this process easily
    - input consists of key presses, mouse movements and on-screen text input
    - output consists of sprite animations, sound effects and on-screen text output
- In contrast, normal programming languages that feature traditional IO can easily be assessed
- Simply provide input through the standard input and check the standard output
- This is not so simple for Scratch

=== What purpose does it have
- Having a way to automatically test Scratch programs allows teachers to grade scratch programs more easily
    - Less time consuming
        - There are programming courses using Scratch that have over 200 students~\cite{itch}
    - may catch easily overlooked errors
    - Gives students an easy way to get feedback for their programs
        - Can detect errors or functionality, which they might have missed
        - Good idea to run a test suite before submitting a solution
- Can also be used outside of courses
    - Can serve as a extension to a self-tutorial
    - Gives feedback to the learner

\section{Contributions}
- This work introduces Whisker as a way to automatically assess the functionality of scratch programs
- Whisker follows a black box testing approach
- Provides a automation tool to control scratch programs from test cases
    - Can simulate input on the program
    - Can query information from the program

- This work shows that automated testing can be used to assess scratch programs
- \textbf{RQ1:} Whisker can facilitate grading of Scratch assignments
    - test results (normal tests and constraint tests) match the results of manual evaluation
    - test results (normal tests and constraint tests) show low rates of false positives and false negatives
    - test results (normal tests and constraint tests) are consistent enough to be used for grading
- \textbf{RQ2:} The testing process can be done entirely in the background independent of testing scenarios (?)
    $\rightarrow$ \textbf{RQ2.1:} Random input can be used for the test
    ($\rightarrow$ The test can involve the user using the program like they normally would)

\section{Outline}
...

%     Overview of the main points.
%     Discuss the research questions.
%     Define the scope of the thesis.
%     Schematic outline for the rest of the thesis.
%
%     Awaken the reader's interest.
%     Should connect with the conclusions, so review / rewrite it in the end.
