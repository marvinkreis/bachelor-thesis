% REMEMBER: Write the thesis from the view of the reader. How would I like to READ the thesis?

\chapter{Introduction}

\section{Overview}
=== What is scratch and why does it matter
- Movement to educational, interactive programming languages in introductory courses
- Scratch is one of the widest spread educational programming language
- Its interactive environment lets users easily build animated interactive programs

=== What purpose does automatic testing have in this context
- Student solutions may need to be evaluated
- Currently no automatic solution to this exists
- Manual assessment of scratch programs is very time consuming
- Interactive nature of scratch programs means much interaction with the program is needed to test it
- Every project has to be opened Individually
- Running the can require many key presses and mouse clicks and can take a long time

- Scratch's IO does not allow to automate this process easily
    - input consists of key presses, mouse movements and on-screen text input
    - output consists of sprite animations, sound effects and on-screen text output
- In contrast, normal programming languages that feature traditional IO can easily be assessed
- Simply provide input through the standard input and check the standard output
- This is not so simple for Scratch

=== What purpose does it have
- Having a way to automatically test Scratch programs allows teachers to grade scratch programs more easily
    - Less time consuming
        - There are programming courses using Scratch that have over 200 students~\cite{itch}
    - may catch easily overlooked errors
    - Gives students an easy way to get feedback for their programs
        - Can detect errors or functionality, which they might have missed
        - Good idea to run a test suite before submitting a solution
- Can also be used outside of courses
    - Can serve as a extension to a self-tutorial
    - Gives feedback to the learner

\section{Contributions}
This thesis introduces a new approach towards dynamic testing of Scratch programs.
The test procedure involves an automation utility, which simulates user input (e.g. mouse movement or key presses) and allows access to information about the sprites and variables of the program.
This allows to test the program in traditional unit-test-like test cases.
Then, we present a way to separate the code from the simulated inputs.
This opens up the possibility to automatically generate user input for tests, which we explore later.

Then, this work introduces Whisker, an implementation of the aforementioned testing utility.
It provides a test-friendly interface to control the Scratch virtual machine through JavaScript methods.
With this implementation, we also explore automated test input generation through a combination of random input and simple static analysis on the Scratch project.

Whisker is evaluated with realistic Scratch programs from different educational courses.
The results provide the following insights:

\mnote{Put statistics here?}
% correlation, avg. number of inconsistencies, avg. coverage
\begin{enumerate}[(1)]
    \item The testing process does not interfere with the execution of the programs under test.
        Scratch programs behave the same during testing and during normal use.
        Specifically, additional computations for testing do not slow the execution of the tested program down.
    \item Test results can be accurate enough to aid in grading Scratch assignments.
        They can closely match the results of manual assessment and show consistent results over multiple test runs.
    \item Automated test input generation is a viable method to control the tested Scratch programs for testing purposes.
        A combination of random input generation and simple static analysis can be used to generate input, which covers a big portion of most Scratch programs' functionality.
\mnote{2 times ''combination of random ...''}
\end{enumerate}

\section{Outline}
...

%     Overview of the main points.
%     Discuss the research questions.
%     Define the scope of the thesis.
%     Schematic outline for the rest of the thesis.
%
%     Awaken the reader's interest.
%     Should connect with the conclusions, so review / rewrite it in the end.
