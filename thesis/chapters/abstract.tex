\begin{abstract}
    Scratch is commonly used to introduce students to the principles of computer programming.
    As some courses in schools and universities are attended by a large number of students, one big disadvantage of Scratch becomes relevant: Grading Scratch assignments is troublesome.
    Scratch's interactive nature makes grading very time-consuming, since many key presses and clicks are required to execute a program.
    At the same time, automating this process is still an open problem, since Scratch's visual and auditory output make automation difficult.
    \parspace

    In order to solve this problem, we implemented Whisker, a program which automates Scratch 3.0's input and output mechanisms to make functional testing for Scratch possible.
    Whisker offers a JavaScript interface, that allows users to simulate input events on programs, and to access Scratch's visual output in the form of sprite attributes and variables.
    Additionally, Whisker offers automated input generation, which can be used to control Scratch programs automatically.
    With this, we explore a property-based approach to testing by defining constraints, that the program under test must hold.
    \parspace

    To evaluate Whisker, we tested a collection of student solutions from a sixth and seventh grade Scratch workshop.
    Our test results closely match scores from independent manual grading, with an average Pearson's correlation coefficient of $0.882$ over ten test executions.
    Furthermore, we evaluated Whisker's automated input generation by measuring its statement coverage on the sample solutions to Code Club's Scratch courses,
    on which was able to achieve a mean statement coverage of $95.25\%$ over ten runs.
\end{abstract}
