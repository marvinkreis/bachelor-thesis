\chapter{Evaluation}

- In this chapter, the Whisker testing utility is evaluated
- The focus of the evaluation lies in comparing the results of automated tests using Whisker to the results of manual assessment
- A collection of real Scratch programs from a student course is tested with three different test suites
- The first two test suites try to assess the projects as well as possible
- The third test suite is used to find out if randomly generated user input can reasonably be used to tests a Scratch project

- Chapter outline

\section{Evaluation Method}
=== Projects (TODO: actually go through the pdfs)
- From a students course
- Two groups, grade 6 and grade 7 $\rightarrow$ realistic for Whisker's purpose
- Multiple exercises to familiarize themselves with Scratch
- Build up to final, most complex, project
- Project has a clear description that stretches about half a DIN A4 page $\rightarrow$ well defined
- [project description]

=== Are these projects suitable for evaluating Whisker?
- Complexity on the higher end of what Whisker is supposed to be used for
- Game Over state makes it hard to assess other properties if the program goes game over because of a bug
    - E.g. one project would often go game over when an apple touches the borders on the sides
    - Well defined $\rightarrow$ tests can be written without looking at the projects (but often problematic, e.g. clones, inexact implementations)
- Students were not aware of the fact, that their projects would be subjected to automated assessment later
    - Students often deliberately changed their programs from the specification (e.g. different fall speeds)
    - In a real scenario (in which Whisker is used for grading) students would have the possibility to test their projects beforehand
    $\rightarrow$ see Chapter 2 - Advantages

=== Tests
- Normal + Constraint Tests
    - Deliberately control the bowl to win the game

- Normal Tests
    - Traditional testing approach: Each test checks one bit of the programs functionality
    - The program gets executed multiple times
    - Better for grading
    - Better for continuous development
    - Example of a normal test (or put it somewhere else?)

- Constraint Tests
    - Different approach: One test executes the program only once and checks defined constraints in the background
    - faster, but less accurate (show difference of average runtime)
    - not everything in this project can be tested in a single run
    - Better for quickly checking the program
    - Example of a constraint test (or put it somewhere else?)

- Random Tests
    - Same (fewer) constraints as the constraint tests
    - Random input instead of deliberately controlling the bowl
    - Random tests only make sense with constraints
        - defining single test cases would be difficult
        - execution would take a long time because most tests are skipped
    - Useful to test the stability of the program (like Android Monkey)
        - Although not on this particular project

\section{Evaluation Setup}
- Chrome / Firefox (/ Electron) version
- JavaScript version
- Whisker version
- Link to tests (and projects?)

\section{Results}
- Manual analysis is used as ground truth

\section{Threats to validity}


TODO: list all specifications and make crosses for the tests and constraints (cross + comment)
\begin{table}
    \centering
    \scriptsize
    \begin{tabular}{ll}
        \toprule
        Tests                          & Constraints\\
        \midrule
        Variable Initialization        & \\
        Bowl Initialization            & \\
        Fruit Initialization           & Fruit Size\\
        Bowl Movement                  & Bowl Movement\\
        Bowl Movement Details          & Bowl Movement Details\\
        Apple Falling                  & \\
        Apple Falling Details          & Apple Falling Details\\
        Banana Falling                 & \\
        Banana Falling Details         & Banana Falling Details\\
        Apple Spawn                    & \\
        Apple Spawn Random X Position  & Apply Spawn Random X Position\\
        Apple Spawn Y Position         & Apple Spawn Y Positions\\
        Banana Spawn                   & \\
        Banana Spawn Random X Position & Banana Spawn Random X Position\\
        Banana Spawn Y Position        & Banana Spawn Y Positions\\
        Only One Apple                 & Only One Apple\\
        Only One Banana                & Only One Banana\\
        Banana Fall Delay              & \\
        Banana Ground Delay            & \\
        Apple Points                   & Apple Points\\
        Apple Game Over                & Apple Game Over\\
        Apple Game Over Message        & Apple Game Over Message\\
        Banana Bowl Points             & Banana Bowl Points\\
        Banana Ground Points           & Banana Ground Points\\
        Banana Ground Message          & Banana Ground Message\\
        Timer Tick                     & Timer Tick\\
        Timer Game Over                & Timer Game Over\\
        Timer Game Over Message        & Timer Game Over Message\\
                                       & Score Change\\
        \bottomrule
    \end{tabular}
    \caption{Tests and corresponding constraints}
    \label{tab:tests_and_corresponding_constraints}
\end{table}


































