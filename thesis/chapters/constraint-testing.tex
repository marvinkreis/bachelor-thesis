\chapter{Property-based Testing with Constraints}
\label{cha:using_constraints_for_flexible_test_inputs}

This section will describe why separating control of the program under test from the test code can be beneficial,
and how this can be achieved by defining constraints, that are checked in the background.

\section{Constraint Tests}
\label{sec:input_independent_constraint_only_tests}

Usually, tests will provide the program under test with inputs and check the resulting outputs.
However, in many cases, a different approach is possible as well.
Tests may use other sources of input and simply observe if the program's output is correct for the input provided by the source.
QuickCheck~\cite{quickcheck} by Claessen et al., for example, uses this principle, to test the correctness of Haskell programs.
In order to do so, tests define conditions, which the program must comply with.
QuickCheck then automatically generates input for the program and checks if the defined conditions hold.
Scratch programs can often be tested in a similar way.
But in order to be able to do this, tests have to be made independent of the simulated input on the program.
This will not just enable us to test with generated input, but with other input sources as well.
For example, the program could be manually controlled by a person, or input could be recorded and played back.
Whisker also offers its own implementation of automated input generation.
\parspace

Using constraints allows us to define conditions, which the program must hold.
Constraints check the program's compliance to the conditions by continually performing assertions throughout the program execution.
Since this is done entirely in the background while the program under test is running, we can define constraints and just let the program run for some time.
This way, it does not matter in what state the program is during its execution, or what inputs it is receiving.
If a condition does not hold, the respective constraint fails.

\section{Testing Procedure}
\label{sec:constraint_testing_procedure}

Figure~\ref{fig:input_independent_testing_procedure} shows a testing procedure, which uses the aforementioned approach.
We will go through each step of the procedure with the aid of an example.
Consider a program with a single sprite, which is supposed to move to the right when, and only when, the right arrow key is pressed.
We want to write a test to check if the sprite's movement works correctly.
For the sake of simplicity we are going to ignore the case of the sprite not being able to move right when touching the right edge of the screen.

\begin{figure}[htpb]
    \centering
    \tikzset{>=latex,
           arrow/.style={draw, -{Latex[length=1.5mm, width=1.5mm]}},
             box/.style={draw, text width=4.3cm, minimum height=0.7cm, text centered, rounded corners},
             num/.style={draw, circle, inner sep=0.6mm, text centered},
               h/.style={fill=blue!10}}

    \begin{tikzpicture}[scale=0.9, every node/.style={scale=0.9}]
        \node[box]    at (0.2, 5.75) (initialize)  {(Give the program some time to initialize)};
        \node[box, h] at (0.2, 4.25) (tracking)    {Setup information tracking};
        \node[box, h] at (0.2, 3.0)  (constraints) {Register constraints};
        \node[box]    at (0.2, 2.0)  (inputs)      {(Simulate Inputs)};
        \node[box]    at (0.2, 1.0)  (run)         {Run the program};
        \node[box, h] at (0.2, 0.0)  (verify)      {Verify tested situation};

        \node[num] at (-2.6, 5.75) (one)   {1};
        \node[num] at (-2.6, 4.25) (two)   {2};
        \node[num] at (-2.6, 3.0)  (three) {3};
        \node[num] at (-2.6, 2.0)  (four)  {4};
        \node[num] at (-2.6, 1.0)  (five)  {5};
        \node[num] at (-2.6, 0.0)  (six)   {6};

        \draw[arrow]
               (0.2,  6.6)
            -- (initialize)
            -- (tracking)
            -- (constraints)
            -- (inputs)
            -- (run)
            -- (verify)
            -- (0.2, -0.8);
    \end{tikzpicture}

    \caption{Constraint testing procedure}
    \label{fig:input_independent_testing_procedure}
\end{figure}

\begin{enumerate}
    \item[(1)] \textbf{Give the program some time to initialize.}
        Before setting up any information tracking or registering constraints we may need to run the program for a short time.
        Otherwise the initialization of the program might be tracked, which will lead to wrong test results.
        \begin{javascriptcode}
            t.runForTime(100);
        \end{javascriptcode}
    \item[(2)] \textbf{Setup information tracking.}
        Since the sprite might not move right in the same execution step the right arrow key is pressed,
        we want to accept slightly delayed movement in the test.
        For this purpose, we track the timestamps when the key starts being pressed,
        when the key was last pressed, and when the sprite last moved.
        We also record if the right arrow key was pressed (and released) at all,
        so we can verify it being pressed (and released) in the end of the test.
        \begin{javascriptcode}
            /* When the last right arrow key press started. */
            let startPressedTime = undefined;
            /* When the right arrow key was last recorded being pressed. */
            let pressedTime = undefined;
            /* When the sprite was last recorded moving right. */
            let movedRightTime = undefined;
            /* If the right arrow key was pressed in the previous step. */
            let previouslyPressed = t.isKeyDown('right arrow');
            /* If the right arrow key was pressed (and released) at all. */
            let pressed = false, released = false;

            const trackRightKeyCb = t.addCallback(() => {
                const currentTime = t.getTotalTimeElapsed();
                if (t.isKeyDown('right arrow')) {
                    pressedTime = currentTime;
                    if (!previouslyPressed) {
                        startPressedTime = currentTime;
                    }
                    previouslyPressed = true;
                    pressed = true;
                } else {
                    previouslyPressed = false;
                    released = true;
                }
            });

            const trackSpriteMoveCb = t.addCallback(() => {
                if (sprite.x > sprite.old.x) {
                    movedRightTime = t.getTotalTimeElapsed();
                }
            });
        \end{javascriptcode}
    \item[(3)] \textbf{Register constraints.}
        Since we tracked when the time when the key was pressed and when the sprite last moved,
        we add a constraint that simply compares the timestamps to determine if the sprite moved at the correct times.
        We chose to allow a delay of $100~\text{ms}$ for the sprite to react to the right arrow key being pressed or being released.
        We also register a second constraint, which makes sure that the sprite only ever stays still or moves to the right,
        but never moves in any other direction.
        \begin{javascriptcode}
            const moveTimingsConstraint = t.addConstraint(() => {
                const currentTime = t.getTotalTimeElapsed();
                if (currentTime > startPressedTime + 100) {
                    t.assert.ok(Math.abs(movedRightTime - pressedTime) <= 100,
                        'Sprite must move right when, and only when, the right arrow key is pressed.');
                }
            });

            const moveDirectionConstraint = t.addConstraint(() => {
                t.assert.ok(sprite.x >= sprite.old.x,
                    'Sprite must only stand still or move right');
                t.assert.ok(sprite.y == sprite.old.y,
                    'Sprite must not move vertically.');
            });
        \end{javascriptcode}
    \item[(4,5)] \textbf{Simulate Inputs, Run the program.}
        Now we may register inputs and run the program.
        How we simulate inputs, or if we perform inputs manually, does not matter for the test, of course.
        \begin{javascriptcode}
            t.setRandomInputInterval(250);
            t.detectRandomInputs();

            await t.runForTime(1000);
        \end{javascriptcode}
    \item[(6)] \textbf{Verify tested situation.}
        If our source of input does not guarantee that the right arrow key gets pressed at all, we risk the chance of a false positive test result,
        since the test never expects the sprite to move right.
        A similar problem occurs when the key is never released.
        Therefore, we want to handle cases in which the right arrow key was either never pressed,
        or was pressed the whole time.
        We may skip the test here, or just disable the constraint and mark it as skipped.
        \begin{javascriptcode}
            t.assume.ok(pressed, 'Right arrow key must be pressed.');
            t.assume.ok(released, 'Right arrow key must be released.');
        \end{javascriptcode}
\end{enumerate}
\parspace

Figure~\ref{fig:normal_input_independent_test_comparison} shows the resulting test code for the example
and compares it to a similar test, which simulates inputs deliberately to check the sprite's movement.
We can easily see that the constraint approach takes quite a bit more code.
But at the same time, the constraint testing procedure is able to scale better.
Once enough tracking is set up, writing constraints becomes easy and requires little code.
At the same time, because the constraints are isolated from the program execution itself,
many constraints can possibly be combined into a single test.
For this purpose, Whisker offers an option to disable failed constraints instead of failing the entire test.
In this case, tests can simply check which constraints failed, and let Whisker generate a test report from them.

\begin{listing}[htpb]
    \centering
    \begin{subfigure}[b]{.40\textwidth}
        \centering
        \begin{minted}[autogobble, breaklines, linenos, fontsize=\tiny, framesep=2mm, frame=lines]{javascript}
            const sprite = t.getSprite('Sprite1');

            /* Give the program some time to initialize. */
            await t.runForTime(100);

            let oldX = sprite.x;

            /* Check that the sprite never moves into an other
             * direction that right. */
            t.addConstraint(() => {
                t.assert.ok(sprite.x >= sprite.old.x,
                    'Sprite must only stand still or move right');
            });

            /* Run without inputs and check that the sprite
             * doesn't move. */

            await t.runForTime(1000);

            t.assert.ok(oldX === sprite.x,
                'Sprite must not move right when no key is pressed.');

            /* Run with the right arrow key being pressed and
             * check that the sprite moves to the right. */

            t.inputImmediate({
                device: 'keyboard',
                key: 'right arrow',
                isDown: true
            });

            await t.runForTime(1000);

            t.assert.ok(oldX < sprite.x,
                'Sprite must move right when right arrow key is pressed.');
        \end{minted}
        \vspace{-\bigskipamount}
        \caption{Normal test}
    \end{subfigure}
    \hspace{.08\textwidth}
    \begin{subfigure}[b]{.50\textwidth}
        \centering
        \begin{minted}[autogobble, breaklines, linenos, fontsize=\tiny, framesep=2mm, frame=lines]{javascript}
            let sprite = t.getSprite('Sprite1');

            /* (1) Give the program some time to initialize. */
            await t.runForTime(100);

            /* When the last right arrow key press started. */
            let startPressedTime = undefined;
            /* When the right arrow key was last recorded being pressed. */
            let pressedTime = undefined;
            /* When the sprite was last recorded moving right. */
            let movedRightTime = undefined;
            /* If the right arrow key was pressed in the previous step. */
            let previouslyPressed = t.isKeyDown('right arrow');
            /* If the right arrow key was pressed (and released) at all. */
            let pressed = false, released = false;

            /* (2) Track when the right arrow key is being pressed,
             * and when the sprite is moving to the right. */

            const trackRightKeyCb = t.addCallback(() => {
                const currentTime = t.getTotalTimeElapsed();
                if (t.isKeyDown('right arrow')) {
                    pressedTime = currentTime;
                    if (!previouslyPressed) {
                        startPressedTime = currentTime;
                    }
                    previouslyPressed = true;
                    pressed = true;
                } else {
                    previouslyPressed = false;
                    released = true;
                }
            });

            const trackSpriteMoveCb = t.addCallback(() => {
                if (sprite.x > sprite.old.x) {
                    movedRightTime = t.getTotalTimeElapsed();
                }
            });

            /* (3) Check if the sprite only moves when the right arrow
             * key was pressed, and if it doesn't move when the key was
             * not pressed. */

            const moveTimingsConstraint = t.addConstraint(() => {
                const currentTime = t.getTotalTimeElapsed();
                if (currentTime > startPressedTime + 100) {
                    t.assert.ok(Math.abs(movedRightTime - pressedTime) <= 100,
                        'Sprite must move right when, and only when, the right arrow key is pressed.');
                }
            }));

            const moveDirectionConstraint = t.addConstraint(() => {
                t.assert.ok(sprite.x >= sprite.old.x,
                    'Sprite must only stand still or move right');
                t.assert.ok(sprite.y == sprite.old.y,
                    'Sprite must not move vertically.');
            }));

            /* (4) Some code, which registers inputs. Or nothing if
             * inputs are done manually. For example, generated input: */
            t.setRandomInputInterval(250);
            t.detectRandomInputs();

            /* (5) Run the program. */
            await t.runForTime(5000);

            /* (6) Check if the right arrow key was pressed at all,
             * and if it was ever released. */
            t.assume.ok(pressed, 'Right arrow key must be pressed.');
            t.assume.ok(released, 'Right arrow key must be released.');
        \end{minted}
        \vspace{-\bigskipamount}
        \caption{Constraint test}
        \label{fig:normal_input_independent_test_comparison_constraint}
    \end{subfigure}
    \caption{Comparison of a normal test and a similar constraint test}
    \label{fig:normal_input_independent_test_comparison}
\end{listing}

\section{Resetting the Program During Tests}
\label{sec:resetting_the_program_during_a_test}

When testing with randomly generated input, programs often need to be reset multiple times inside of one test execution,
because the test could get stuck in some state of the program.
\parspace

To reset a Scratch program, we may simply press the green flag button again,
but we also need to consider the tracked information as well as the registered callbacks and constraints when resetting the program under test.
Any per-run information we tracked has to be cleared, and callbacks as well as constraints have to disabled before resetting the program.
After resetting the program, callbacks and constraints have to be enabled again.
Also, if Whisker was configured not to fail the test when a constraint fails, then the test needs to check which constraints failed before disabling them
and then only enable constraints which were active before.
Figure~\ref{fig:resetting_the_program_under_test_procedure} shows a procedure that may be used to reset the program under test.
This procedure replaces step $5$ (Run the program) in the constraint testing procedure (see Figure~\ref{fig:input_independent_testing_procedure}).
Additionally the code example in Listing~\ref{fig:resetting_the_program_under_test_example}
implements this procedure for the example from the previous section.
\parspace

Sometimes, information may need to be tracked across resets in order to determine what events occurred in the program during the whole test execution.
This information may be needed to rule out constraints, whose situation never occurred, after the test execution.
An example for this are the variables \texttt{pressed} and \texttt{released} in Listing~\ref{fig:normal_input_independent_test_comparison_constraint}.
\parspace


\begin{figure}[htpb]
    \centering
    \begin{subfigure}[m]{.3\textwidth}
        \centering
        \tikzset{>=latex,
               arrow/.style={draw, -{Latex[length=1.5mm, width=1.5mm]}},
                 box/.style={draw, text width=4.3cm, minimum height=0.7cm, text centered, rounded corners},
                 num/.style={draw, circle, inner sep=0.6mm, text centered},
                   h/.style={fill=blue!10}}

        \begin{tikzpicture}[scale=0.9, every node/.style={scale=0.9}]
            \node[box, h] at (0.2,  6.75) (suspendcb)  {Suspend Information Tracking};
            \node[box, h] at (0.2,  5.50) (suspendcn)  {Suspend Constraints};
            \node[box, h] at (0.2,  4.25) (resetinfo)  {Reset per-run information};
            \node[box]    at (0.2,  2.25) (greenflag)  {Press the green flag (and give the program some time to Initialize)};
            \node[box]    at (0.2,  0.75) (spriteref)  {Get new sprites};
            \node[box, h] at (0.2, -0.75) (activatecb) {Activate Information Tracking};
            \node[box, h] at (0.2, -2.00) (activatecn) {Activate Constraints};
            \node[box]    at (0.2, -3.00) (run)        {Run the program};


            \node[num] at (-2.6,  6.75) (one)   {1};
            \node[num] at (-2.6,  5.50) (two)   {2};
            \node[num] at (-2.6,  4.25) (three) {3};
            \node[num] at (-2.6,  2.25) (four)  {4};
            \node[num] at (-2.6,  0.75) (five)  {5};
            \node[num] at (-2.6, -0.75) (six)   {6};
            \node[num] at (-2.6, -2.00) (seven) {7};
            \node[num] at (-2.6, -3.00) (eight) {8};

            \draw[arrow]
                   ( 0.2,  7.6)
                -- (suspendcb)
                -- (suspendcn)
                -- (resetinfo)
                -- (greenflag)
                -- (spriteref)
                -- (activatecb)
                -- (activatecn)
                -- (run)
                -- ( 0.2, -3.8);

            % \draw[-, rounded corners]
            %        ( 0.2, -2.8)
            %     -- (-3.4, -2.8)
            %     -- (-3.4,  7.7)
            %     -- ( 0.2,  7.7);
        \end{tikzpicture}
        \caption{Procedure for resetting the program under test}
        \label{fig:resetting_the_program_under_test_procedure}
    \end{subfigure}%
    \hspace{.13\textwidth}%
    \begin{subfigure}[m]{.55\textwidth}
        \centering
        \begin{minted}[autogobble, breaklines, linenos, fontsize=\scriptsize, framesep=2mm, frame=lines]{javascript}
            for (let i = 0; i < 5; i++) {
                const constraints = [moveTimingsConstraint,
                    moveDirectionConstraint];

                /* (1) Suspend information tracking. */
                trackRightKeyCb.disable();
                trackSpriteMoveCb.disable();

                /* (2) Suspend constraints.
                 * Get a list active constraints first.
                 * (Constraints that did not fail) */
                const ac = constraints.filter(c => c.isActive());
                for (const constraint of ac) {
                    constraint.disable();
                }

                /* (3) Reset per-run information
                   (keep values of 'pressed' and 'released'). */
                startPressedTime = undefined;
                pressedTime = undefined;
                movedRightTime = undefined;
                previouslyPressed = t.isKeyDown('right arrow');

                /* (4) Press the green flag and give the program
                   some time to Initialize. */
                t.greenFlag();
                await t.runForTime(100);

                /* (5) Get new sprites. */
                sprite = t.getSprite('Sprite1');

                /* (6) Activate information tracking. */
                trackRightKeyCb.enable();
                trackSpriteMoveCb.enable();

                /* (7) Activate constraints. */
                for (const constraint of ac) {
                    constraint.enable();
                }

                /* (8) Run the program. */
                await t.runForTime(2000);
            }
        \end{minted}
        \vspace{-\bigskipamount}
        \caption{Example code for resetting the program under test (extends Listing~\ref{fig:normal_input_independent_test_comparison_constraint})}
        \label{fig:resetting_the_program_under_test_example}
    \end{subfigure}

    \caption{Procedure and example code for resetting the program under test}
    \label{fig:resetting_the_program_under_test}
\end{figure}


% Register Callbacks for Information Tracking (per run for constraints and across runs to rule out constraints in the end)
% Register Constraints
%
% repeat:
%     Suspend Information Tracking Callbacks
%     Suspend Constraints
%
%     Reset Per Run Information
%
%     Press Green Flag
%     Run the Program For Short Time to Initialize It
%
%     Activate Information Tracking Callbacks
%     Activate Constraints
%
%     Run the Program
%
% Check Across-Run Information to Disable Constraints, whose situation to check did not occur
% Generate test report


% \begin{figure}[htpb]
%     \centering
%     \begin{subfigure}[b]{.45\textwidth}
%         \centering
%         \tikzset{>=latex,
%                  box/.style={draw, text width=4.3cm, minimum height=0.7cm, text centered, rounded corners},
%                  num/.style={draw, circle, inner sep=0.6mm, text centered},
%                    h/.style={fill=blue!10}}
%
%         \begin{tikzpicture}
%             \node[box] at ( 0.2,  3.0) (run)        {Run the program};
%             \node[box] at ( 0.2,  2.0) (inputs)     {Simulate inputs};
%             \node[box] at ( 0.2,  1.0) (checks)     {Perform checks};
%
%             \node[num] at (-2.6,  3.0) (one)   {1};
%             \node[num] at (-2.6,  2.0) (two)   {2};
%             \node[num] at (-2.6,  1.0) (three) {3};
%
%             \draw[->]
%                    ( 0.2,  4.8)
%                 -- (run)
%                 -- (inputs)
%                 -- (checks)
%                 -- ( 0.2, -1.0);
%
%             \draw[shorten >= 2pt, rounded corners, dashed, ->]
%                    ( 0.2,  0.0)
%                 -- (-3.4,  0.0)
%                 -- (-3.4,  4.0)
%                 -- ( 0.2,  4.0);
%         \end{tikzpicture}
%
%         \caption{Normal Test Procedure}
%         \label{fig:normal_test_procedure}
%     \end{subfigure}
%     \begin{subfigure}[b]{.45\textwidth}
%         \centering
%         \tikzset{>=latex,
%                  box/.style={draw, text width=4.3cm, minimum height=0.7cm, text centered, rounded corners},
%                  num/.style={draw, circle, inner sep=0.6mm, text centered},
%                    h/.style={fill=blue!10}}
%
%         \begin{tikzpicture}
%             \node[box] at ( 0.2,  4.25) (tracking)    {Setup tracking of information};
%             \node[box] at ( 0.2,  3.0)  (constraints) {Register constraints};
%             \node[box] at ( 0.2,  2.0)  (run)         {Run the program};
%             \node[box] at ( 0.2,  1.0)  (inputs)      {(Simulate Inputs)};
%             \node[box] at ( 0.2,  0.0)  (filter)      {Filter constraints};
%
%             \node[num] at (-2.6,  4.25) (one)   {1};
%             \node[num] at (-2.6,  3.0)  (two)   {2};
%             \node[num] at (-2.6,  2.0)  (three) {3};
%             \node[num] at (-2.6,  1.0)  (four)  {4};
%             \node[num] at (-2.6,  0.0)  (five)  {5};
%
%             \draw[->]
%                    (tracking)
%                 -- (constraints)
%                 -- (run)
%                 -- (inputs)
%                 -- (filter)
%                 -- ( 0.2, -1.0);
%         \end{tikzpicture}
%
%         \caption{Constraint-only Test Procedure}
%         \label{fig:constraint_only_test_procedure}
%     \end{subfigure}
%     \caption{Comparison of the procedure of normal tests and constraint-only tests}
%     \label{fig:comparison_of_the_procedure_of_normal_tests_and_constraint_only_tests}
% \end{figure}
